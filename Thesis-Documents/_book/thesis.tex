% This is the Reed College LaTeX thesis template. Most of the work
% for the document class was done by Sam Noble (SN), as well as this
% template. Later comments etc. by Ben Salzberg (BTS). Additional
% restructuring and APA support by Jess Youngberg (JY).
% Your comments and suggestions are more than welcome; please email
% them to cus@reed.edu
%
% See http://web.reed.edu/cis/help/latex.html for help. There are a
% great bunch of help pages there, with notes on
% getting started, bibtex, etc. Go there and read it if you're not
% already familiar with LaTeX.
%
% Any line that starts with a percent symbol is a comment.
% They won't show up in the document, and are useful for notes
% to yourself and explaining commands.
% Commenting also removes a line from the document;
% very handy for troubleshooting problems. -BTS

% As far as I know, this follows the requirements laid out in
% the 2002-2003 Senior Handbook. Ask a librarian to check the
% document before binding. -SN

%%
%% Preamble
%%
% \documentclass{<something>} must begin each LaTeX document
\documentclass[12pt,twoside]{reedthesis}
% Packages are extensions to the basic LaTeX functions. Whatever you
% want to typeset, there is probably a package out there for it.
% Chemistry (chemtex), screenplays, you name it.
% Check out CTAN to see: http://www.ctan.org/
%%
\usepackage{graphicx,latexsym}
\usepackage{amsmath}
\usepackage{amssymb,amsthm}
\usepackage{longtable,booktabs,setspace}
\usepackage{chemarr} %% Useful for one reaction arrow, useless if you're not a chem major
\usepackage[hyphens]{url}
% Added by CII
\usepackage{hyperref}
\usepackage{lmodern}
\usepackage{float}
\floatplacement{figure}{H}
% End of CII addition
\usepackage{rotating}

% Next line commented out by CII
%%% \usepackage{natbib}
% Comment out the natbib line above and uncomment the following two lines to use the new
% biblatex-chicago style, for Chicago A. Also make some changes at the end where the
% bibliography is included.
%\usepackage{biblatex-chicago}
%\bibliography{thesis}


% Added by CII (Thanks, Hadley!)
% Use ref for internal links
\renewcommand{\hyperref}[2][???]{\autoref{#1}}
\def\chapterautorefname{Chapter}
\def\sectionautorefname{Section}
\def\subsectionautorefname{Subsection}
% End of CII addition

% Added by CII
\usepackage{caption}
\captionsetup{width=5in}
% End of CII addition

% \usepackage{times} % other fonts are available like times, bookman, charter, palatino

% Syntax highlighting #22
  \usepackage{color}
  \usepackage{fancyvrb}
  \newcommand{\VerbBar}{|}
  \newcommand{\VERB}{\Verb[commandchars=\\\{\}]}
  \DefineVerbatimEnvironment{Highlighting}{Verbatim}{commandchars=\\\{\}}
  % Add ',fontsize=\small' for more characters per line
  \usepackage{framed}
  \definecolor{shadecolor}{RGB}{248,248,248}
  \newenvironment{Shaded}{\begin{snugshade}}{\end{snugshade}}
  \newcommand{\KeywordTok}[1]{\textcolor[rgb]{0.13,0.29,0.53}{\textbf{#1}}}
  \newcommand{\DataTypeTok}[1]{\textcolor[rgb]{0.13,0.29,0.53}{#1}}
  \newcommand{\DecValTok}[1]{\textcolor[rgb]{0.00,0.00,0.81}{#1}}
  \newcommand{\BaseNTok}[1]{\textcolor[rgb]{0.00,0.00,0.81}{#1}}
  \newcommand{\FloatTok}[1]{\textcolor[rgb]{0.00,0.00,0.81}{#1}}
  \newcommand{\ConstantTok}[1]{\textcolor[rgb]{0.00,0.00,0.00}{#1}}
  \newcommand{\CharTok}[1]{\textcolor[rgb]{0.31,0.60,0.02}{#1}}
  \newcommand{\SpecialCharTok}[1]{\textcolor[rgb]{0.00,0.00,0.00}{#1}}
  \newcommand{\StringTok}[1]{\textcolor[rgb]{0.31,0.60,0.02}{#1}}
  \newcommand{\VerbatimStringTok}[1]{\textcolor[rgb]{0.31,0.60,0.02}{#1}}
  \newcommand{\SpecialStringTok}[1]{\textcolor[rgb]{0.31,0.60,0.02}{#1}}
  \newcommand{\ImportTok}[1]{#1}
  \newcommand{\CommentTok}[1]{\textcolor[rgb]{0.56,0.35,0.01}{\textit{#1}}}
  \newcommand{\DocumentationTok}[1]{\textcolor[rgb]{0.56,0.35,0.01}{\textbf{\textit{#1}}}}
  \newcommand{\AnnotationTok}[1]{\textcolor[rgb]{0.56,0.35,0.01}{\textbf{\textit{#1}}}}
  \newcommand{\CommentVarTok}[1]{\textcolor[rgb]{0.56,0.35,0.01}{\textbf{\textit{#1}}}}
  \newcommand{\OtherTok}[1]{\textcolor[rgb]{0.56,0.35,0.01}{#1}}
  \newcommand{\FunctionTok}[1]{\textcolor[rgb]{0.00,0.00,0.00}{#1}}
  \newcommand{\VariableTok}[1]{\textcolor[rgb]{0.00,0.00,0.00}{#1}}
  \newcommand{\ControlFlowTok}[1]{\textcolor[rgb]{0.13,0.29,0.53}{\textbf{#1}}}
  \newcommand{\OperatorTok}[1]{\textcolor[rgb]{0.81,0.36,0.00}{\textbf{#1}}}
  \newcommand{\BuiltInTok}[1]{#1}
  \newcommand{\ExtensionTok}[1]{#1}
  \newcommand{\PreprocessorTok}[1]{\textcolor[rgb]{0.56,0.35,0.01}{\textit{#1}}}
  \newcommand{\AttributeTok}[1]{\textcolor[rgb]{0.77,0.63,0.00}{#1}}
  \newcommand{\RegionMarkerTok}[1]{#1}
  \newcommand{\InformationTok}[1]{\textcolor[rgb]{0.56,0.35,0.01}{\textbf{\textit{#1}}}}
  \newcommand{\WarningTok}[1]{\textcolor[rgb]{0.56,0.35,0.01}{\textbf{\textit{#1}}}}
  \newcommand{\AlertTok}[1]{\textcolor[rgb]{0.94,0.16,0.16}{#1}}
  \newcommand{\ErrorTok}[1]{\textcolor[rgb]{0.64,0.00,0.00}{\textbf{#1}}}
  \newcommand{\NormalTok}[1]{#1}

% To pass between YAML and LaTeX the dollar signs are added by CII
\title{My Final College Paper}
\author{Emerson H. Webb}
% The month and year that you submit your FINAL draft TO THE LIBRARY (May or December)
\date{May 2018}
\division{Mathematics and Natural Sciences}
\advisor{Advisor F. Name}
\institution{Reed College}
\degree{Bachelor of Arts}
%If you have two advisors for some reason, you can use the following
% Uncommented out by CII
% End of CII addition

%%% Remember to use the correct department!
\department{Mathematics}
% if you're writing a thesis in an interdisciplinary major,
% uncomment the line below and change the text as appropriate.
% check the Senior Handbook if unsure.
%\thedivisionof{The Established Interdisciplinary Committee for}
% if you want the approval page to say "Approved for the Committee",
% uncomment the next line
%\approvedforthe{Committee}

% Added by CII
%%% Copied from knitr
%% maxwidth is the original width if it's less than linewidth
%% otherwise use linewidth (to make sure the graphics do not exceed the margin)
\makeatletter
\def\maxwidth{ %
  \ifdim\Gin@nat@width>\linewidth
    \linewidth
  \else
    \Gin@nat@width
  \fi
}
\makeatother

\renewcommand{\contentsname}{Table of Contents}
% End of CII addition

\setlength{\parskip}{0pt}

% Added by CII

\providecommand{\tightlist}{%
  \setlength{\itemsep}{0pt}\setlength{\parskip}{0pt}}

\Acknowledgements{
I want to thank a few people.
}

\Dedication{
You can have a dedication here if you wish.
}

\Preface{
This is an example of a thesis setup to use the reed thesis document
class (for LaTeX) and the R bookdown package, in general.
}

\Abstract{
The preface pretty much says it all. \par

Second paragraph of abstract starts here.
}

	\usepackage{tikz, algorithm}
	\usepackage[noend]{algpseudocode}
% End of CII addition
%%
%% End Preamble
%%
%

\usepackage{amsthm}
\newtheorem{theorem}{Theorem}[chapter]
\newtheorem{lemma}{Lemma}[chapter]
\theoremstyle{definition}
\newtheorem{definition}{Definition}[chapter]
\newtheorem{corollary}{Corollary}[chapter]
\newtheorem{proposition}{Proposition}[chapter]
\theoremstyle{definition}
\newtheorem{example}{Example}[chapter]
\theoremstyle{definition}
\newtheorem{exercise}{Exercise}[chapter]
\theoremstyle{remark}
\newtheorem*{remark}{Remark}
\newtheorem*{solution}{Solution}
\begin{document}

% Everything below added by CII
  \maketitle

\frontmatter % this stuff will be roman-numbered
\pagestyle{empty} % this removes page numbers from the frontmatter
  \begin{acknowledgements}
    I want to thank a few people.
  \end{acknowledgements}
  \begin{preface}
    This is an example of a thesis setup to use the reed thesis document
    class (for LaTeX) and the R bookdown package, in general.
  \end{preface}
  \hypersetup{linkcolor=black}
  \setcounter{tocdepth}{2}
  \tableofcontents

  \listoftables

  \listoffigures
  \begin{abstract}
    The preface pretty much says it all. \par
    
    Second paragraph of abstract starts here.
  \end{abstract}
  \begin{dedication}
    You can have a dedication here if you wish.
  \end{dedication}
\mainmatter % here the regular arabic numbering starts
\pagestyle{fancyplain} % turns page numbering back on

\chapter*{Introduction}\label{introduction}
\addcontentsline{toc}{chapter}{Introduction}

\chapter{Introduction to Trees, Random Forests, and the
Bootstrap}\label{rmd-basics}

This thesis is about understanding the variance in variable importance
measures of random forests. Random Forests are a statistical learning
algorithm developed by Leo Breiman and his collaborators in the early
2000's that leverages bagging and CART (Classification and Regression
Trees) methodology to produce quite good predictions of an underlying
classification or regression surface. The predictive capability of
random forests is quite impressive and much random forest literature
centers on exploring why random forests are effective. For this thesis
we are interested in the use of random forests for statistical
inference. More specifically we are interested in the use of random
forest variable importance measures for statistical inference.

Machine learning approaches to regression can often produce models with
good predictive accuracy compared to parametric modelling approaches.
However, these machine learning algorithms often lack in the
interpretability and inferential capabilities of more traditional
statistical modelling. While predictive questions are indeed important
in science and other applications, many scientific and statistical
questions are concerned with description and inference. These questions
could be about a causal mechanism, which hypothesis is correct
concerning some phenomenon, and which are the important factors. The
mechanisms that machine learning techniques use to form predictions on
the data are often complicated and not easily amenable to mathematical
analysis, but we believe one of Breiman's points in his 2001 paper
``Statistical Modelling: Two Cultures'' is that statisticians ought to
work to develop inferential tools for machine learning techniques.
Predictive accuracy is a good metric to measure how a model fits to the
data, and without a method that can produce predictively accurate
estimates of the underlying data mechanism, inferential conclusions seem
less justifiable. Throughout this thesis we will be interested in both
the regression and classification settings as random forests can handle
response variables of both types. When the need arises to delineate
between the setting we are working in, we will be explicit. Furthermore,
we will be assuming that for the data at hand, there exists an
underlying regression function \(Y=f(X)+\varepsilon\) where \(f(X)\) is
the function and \(\varepsilon\) is generally assumed to be Gaussian
error, but may have some other error structure.

\section{CART}\label{cart}

We begin our discussion of CART (Classification and Regression Trees)
trees by considering the following problem. Suppose we have data
generated from the piecewise function \(Y=f(X)+\varepsilon\) where {[}at
this point insert a very non-linear piecewise function.{]} with Gaussian
error \(\varepsilon\). One approach to fitting a model to this data
would be to fit a linear regression model. However, the particular form
of the data is not well suited for linear regression due to the
non-linearity of the data. Another approach we might try to take is as
follows. We might try to find a method that can fit a regression line to
each piecewise componenet of the data. One such method is to fit CART
trees. We fit a CART tree to the data and perform cost-complexity
pruning to output the model shown in figure {[}figure not yet made{]}.
In this example, we produced a regression surface which was able to take
into account the piecewise, non-linear nature of the data and produce a
reasonable estimate of the underlying data generating mechanism.

The CART methodology allows us to fit a model that can take into account
non-linear regression or classification surfaces. The basic idea of CART
is that if we can split the predictors into roughly homogenous
partitions, or into partitions which minimize predictive error, then we
can fit a simple model to predict the response of each partition. CART
trees were first introduced by Breiman et al. in 1984 and are a flexible
method, capable of handling classification and regression settings. CART
is a product of computational statistics, it's development influenced by
a data-centric algorithmic approach distinct from the models of
classical statistics influenced by the contingency of small samples.

More formally, suppose we have a training data set
\(Z=\{Z_1,\ldots,Z_n\}\), where \(Z_i=(X_i,Y_i)\) is an \(p+1\)
dimensional vector in \(\mathbb{R}^{p+1}\). Here we have \(X_i\) is a
\(d\)-dimensional predictor variable and \(Y_i\) is the response. In
particular, we can consider \(Z\) to be a \(n\times (p+1)\) array where
the rows are observations and the columns are the response and predictor
variables. CART works by partitioning the data through binary recursive
splits via optimizing some loss function. CART trees are called trees
because partitioning the data through binary recursive splits forms a
tree-like structure. We adopt notation evocative of this tree structure.
Any subset of the training data \(Z\) is called a node while the entire
data set \(Z\) is called the root node. Nodes are of two kinds: they are
either terminal (sometimes called leaves) or not terminal. Non-terminal
nodes are nodes that are split on in the tree growing process, while
terminal nodes are nodes which are not split on. Generally, a node is a
terminal node if some stopping rule is reached. Note that terminal nodes
form a partition of the training data \(Z\).

We aim to grow a tree with roughly homogenous terminal nodes. As Breiman
et al. (1984) note, there are several factors that we need to consider:
\begin{enumerate}
        \item How to select splits of the data.
        
        \item When to stop splitting the data.
        
        \item How to assign classes or values to terminal nodes. 
    \end{enumerate}
CART constructs trees by decreasing the nodal impurity of a node \(t\).
Nodal impurity is defined through some nodal impurity measure
\(i(s,t)\), usually the GINI index in the classification setting and the
residual sum of squares in the regression setting. We present the
regression setting and then the classification setting. If we have a
continuous response \(Y\), then we could try to predict \(Y\) by
partitioning the data using the decision tree structure and predicting
that points that fall within a particular partition will take on the
average response on that partition. To choose the best binary split on
the data, we need to search across the splitting variables and splitting
points for the split which maximizes the reduction in the RSS between
the parent node and daughter nodes. In particular, we want to find the
split which maximizes \(RSS_l(j,s)+RSS_r(j,s)\) where \(j\) is the
proposed splitting variable and \(s\) is the proposed splitting point,
and \(RSS_l\) is the RSS of the proposed left node and \(RSS_r\) is the
RSS of the proposed right node. The assignment of response within a node
is given by the average of the response. We present the following
algorithm.
\begin{algorithm}
        \caption{Construction of Regression tree}\label{regression tree}
        \begin{algorithmic}[1]
            \While {minimum node size not reached}
                \For {each node $t$ }
                    \For {$j=1,\ldots,p$ and $s=1,\ldots,n$ }
                    \State Compute $RSS_l(j,s)+RSS_r(j,s)$.
                    \EndFor
                    \State Pick the $(j,s)$ which maximizes $RSS_l(j,s)+RSS_r(j,s)$.
                    \State Split the current node into left and right nodes according to $(j,s)$.
                    \State Compute left and right averages, $\text{ave}(y_i|x_i\in t_l)$ and $\text{ave}(y_i|x_i\in t_r)$, respectively. 
                \EndFor
                \State Output the tree $T$.
            \EndWhile
        \end{algorithmic}
    \end{algorithm}
The construction of classification trees is similar to the construction
of regression trees except a different impurity measure must be used.
For the node \(t\) containing \(N_t\) observations, let
\[\hat{p}_{tk}=\frac{1}{N_t}\sum_{x_i\in t} \mathbb{I}(y_i=k)\] where
\(k=0\) or \(k=1\). Then \(\hat{p}_{tk}\) measures the proportion of
observations of class \(k\) in node \(t\). Using the two-class example,
there are several impurity measures available in the classification
setting. The most common measure is perhaps the Gini index defined by
\[2p(1-p).\] Other options include misclassification error
\[1-\max(p,1-p),\] and cross-entropy \[-p\log(p)-(1-p)\log(1-p).\] If
\(t_L\) and \(t_R\) are left and right nodes proposed under the split,
let \(\hat{p}_{tL}\) and \(\hat{p}_{tR}\) be proportion of observations
falling into \(t_L\) and \(t_R\), respectively. Denote the Gini index of
the left node \(t_L\) by \(G_{L}\) and denote the Gini index of the
right node \(t_R\) by \(G_{R}\). Then our splitting criterion is to seek
the spliting variable and splitting point which minimizes
\[\hat{p}_{tL}G_L+\hat{p}_{tR}G_R.\] The case for misclassification rate
and cross-entropy as the impurity measures is similar.

Some issues with CART trees include overfitting and variability. While
there is a stopping criterion for growing the CART trees, often times
naively growing a tree can result in overfitting to the data. One method
of alleviating overfitting is to employ cost-complexity pruning, which
searches for an optimal tree that balances fitting a good tree model
with overfitting to the data. We do not go into details here, but the
point the reader to Friedman et al. (2009). A larger issue with trees is
their variability to small perturbations in the data. {[}Provide
computational example.{]} Allowing the data to vary even a bit can
result in a different tree structure upon refitting. As a statistical
learning method, this means that CART trees are not robust and suffer
from high variance, even when constructed using cost-complexity pruning.
To get around this issue of variability the best solution is to employ
bagging and use the random forest algorithm.

\section{Bootstrap}\label{bootstrap}

We feel it is worth introducing some notation and theory for the
bootstrap. In this line, for the most part we follow notation from Efron
and Tibshirani (1993). As is well-known, the bootstrap is a resampling
method first introduced by Efron in 1979 with links to earlier work of
Tukey and Quenouille on the jackknife. It goes something like this. If
we have a dataset \(D\) of size \(n\), then it is generally infeasible
to obtain more data from the underlying data generating mechanism.
Assuming that the data set is not too heavily distributed in the tails
and that the sample we have is representative of the population, we can
independently resample \(n\) points from \(D\) to form the bootstrap
resampled dataset \(D^*\). We repeat this procedure \(B\) many times
with \(B\) generally quite large. We can then proceed to use the many
\(D^*\) we generated to obtain estimates, standard errors, and
confidence intervals of whichever parameter is of interest. {[}Note: the
bootstrap discussed here is called the non-parametric bootstrap. There
is also the parametric bootstrap, but we do not go into that method
here{]}.

What is going on? If we obtain an i.i.d. random sample of size \(n\)
from a probability distribution \(F\), then we can form the empirical
distribution function \(\hat{F}\) which places a probability of equal
mass \(\frac{1}{n}\) of seeing any value of \(x_i\) for
\(i=1,\ldots,n\). We do not want to go too deeply into bootstrap theory
in this introduction, but we would like to mention the plug-in principle
underlying applications of the bootstrap, as well as variance
estimation.

In statistics, we are interested in obtaining information about
parameters defined on a probability distribution \(F\) by dealing with
statistics defined on the EDF \(\hat{F}\). Often a parameter is a
function of the probability distribution: \[\theta=t(F)\] where in this
context \(t\) denotes the functional form of the parameter. Often the
probability distribution \(F\) is unknown or too complex to deal with
directly, so we try to form estimates. One estimate we could form of the
parameter \(\theta=t(F)\) is the plug-in estimate (or plug-in statistic,
or functional statistic), which is defined to be
\[\hat{\theta}=t(\hat{F}).\] It can be shown under suitable conditions,
that the functional statistic \(t(\hat{F})\) is the nonparametric
maximum likelihood estimate of \(t(F)\) (Efron and Tibshirani).
Therefore there is good theoretical ground for proceeding with
estimating parameters using the bootstrap.

The usefulness of the bootstrap extends beyond estimating functional
statistics. Using the bootstrap, we can estimate the variance of the
functional statistic using several methods. The most ubiquitous is
perhaps the bootstrap estimate of the standard error. Say we have drawn
\(B\) many bootstrap samples of a dataset \(Z\) to obtain \(Z_b^*\) for
\(b=1,\ldots,B\). We compute an estimate \(\hat{\theta}=s(Z)\) of
\(\theta=t(F)\), where Efron and Tibshirani emphasize that \(s(Z)\) is
not necessarily the plug-in estimate \(t(\hat{F})\). To emphasize that
the estimates computed from \(Z_b^*\) are bootstrap replicates of
\(\hat{\theta}\), we write \[\hat{\theta}_b^*=s(Z_b^*).\] In some
settings we write \(\hat{\theta}^*=s(Z^*)\) when we do not care as much
about the actual subscripting from the bootstrap and just want to
emphasize the fact we are estimating \(\hat{\theta}\) using bootstrap
replicate.

Denote the standard error of the statistic \(\hat{\theta}\) by
\(se_F(\hat{\theta})\). Following Efron and Tibshirani, the bootstrap
estimate of the \(se_F(\hat{\theta})\) is a plug-in estimate of the
standard error of \(\hat{\theta}\) replacing the distribution \(F\) in
the subscript by the corresponding empirical distribution \(\hat{F}\)
and is given by \(se_{\hat{F}}(\hat{\theta}^*)\). Efron and Tibshirani
call \(se_{\hat{F}}(\hat{\theta})\) the ideal bootstrap estimate of the
standard error of \(\hat{\theta}\). Beyond simple examples like the
mean, the exact form of \(se_{\hat{F}}\) is often too difficult to
compute exactly, but the bootstrap can be used to compute a good
estimate of \(se_{\hat{F}}\). Namely, once we have drawn
\(Z_1^*, \ldots, Z_B^*\) independent bootstrap samples from \(Z\) and
compute the \(B\) many bootstrap replicates
\(\hat{\theta}_b^*=s(Z_b^*)\), then the estimate of the standard error
\(se_{F}(\hat{\theta})\) is given by
\[\widehat{se}_B=\left(\frac{1}{B-1}\sum_{b=1}^B (\hat{\theta}_b^*-\bar{\theta}^*)\right)^{1/2},\]
where \(\bar{\theta}^*=\frac{1}{B}\sum_{b=1}^B\hat{\theta}_b^*\) is the
sample mean of the bootstrap replicates. Note \(\widehat{se}_B\) is the
familiar formula for the standard deviation of a sample.

There are several other estimates of the standard error commonly used
when exact forms of the standard error are unavailable or infeasible.
The first of these is the jackknife. As a computational tool, the
jackknife takes a different approach from the bootstrap. Say for the
parameter \(\theta\), we have an estimate \(\hat{\theta}=s(Z)\). Define
the \(i\)-th ``leave one out'' sample
\(Z_{(i)}=(Z_1,\ldots,Z_{i-1},Z_{i+1},\ldots,Z_{n})\). Corresponding to
\(Z_{(i)}\) we can form \(\hat{\theta}_{(i)}=s(Z_{(i)})\), the \(i\)th
jackknife replicate of the estimator \(\hat{\theta}\). Define
\[\hat{\theta}_{(\cdot)}=\frac{1}{n}\sum_{i=1}^n \hat{\theta}_{(i)}\]
which is akin to the mean of the jackknife replications. From here we
can obtain the jackknife estimate of the standard error, which is
defined by
\[\widehat{se}_\text{jack} =\left(\frac{n-1}{n}\sum_{i=1}^n (\hat{\theta}_{(i)}-\hat{\theta}_{(\cdot)})^2\right)^{1/2}.\]
The idea of the jackknife is that absent new realizations of the data, a
simple procedure we can undertake is to see the effect on the accuracy
of the estimator when we remove each observation sequentially. In
particular, we could consider the jackknife as forming \(n\) many
datasets of size \(n-1\) with the \(i\)th observation removed. The
jackknife may be easier to compute than the bootstrap, but as the
jackknife only uses limited information about the \(n\) observations and
does not simulate additional full datasets like the bootstrap, overall
it is less efficient than the bootstrap. The jackknife is related to the
bootstrap in the sense that the jackknife is a linear approximation to
the bootstrap. As Efron and Tibshirani explain, for linear statistics
the jackknife is as efficient as the bootstrap. However, for highly
non-linear statistics (of which it turns out random forests are), it
turns out that jackknife estimation is very inefficient compared to the
bootstrap. There are other methods of estimating the standard error such
as the Infinitesimal Jackknife and Non-parametric delta method, but we
will address those methods in later chapters.

The bootstrap is a flexible method and applications of the bootstrap
extend beyond simply sampling pairs of observations from the the data
\(Z=\{Z_1=(X_1,Y_1), Z_2=(X_2,Y_2),\ldots, Z_n=(X_n,Y_n)\}\). In
particular, there are model-based approaches to the bootstrap that are
often useful. A particularly important example for this thesis is
bootstrapping the least squares linear regression model. In least
squares, we model the response \(Y\) observed at the vector \(X\) as
being given by \(\mathbb{E}(Y|X)=X^T\beta+\varepsilon\), where
\(X=(1,X_1,\ldots,X_p)^T\) and
\(\beta=(\beta_0,\beta_1,\ldots,\beta_p)\) and \(\varepsilon\) is an
error term of length \(n\). In practice, the vector of coefficients
\(\beta\) is unknown, so we form an estimate \(\hat{\beta}\) through
applying least squares regression. After applying least squares. For a
point \(X_j\), the least squares estimate of \(Y_j\) is given by
\(\hat{Y_j}=X_j\hat{\beta}\). The raw residuals are defined by
\(e_j=Y_j-\hat{Y_j}\) with the estimate of the variance \(\sigma^2\) is
given by \[s^2=\frac{1}{n-2}\sum_{j=1}^n e_j^2\].

There are a couple approaches we could consider in applying the
bootstrap to linear regression. The conceptually simpler method is to
resample cases \((X_i,Y_i)\) to generate bootstrap replications of
\(Z\), \(Z^*=\{(X_1^*,Y_1^*),\ldots,(X_n^*,Y_n^*)\}\) where the
asterisks denote that we have resampled from the original data. With
\(B\) many such datasets \(Z^*\), we can use the following algorithm
adapted from Davison and Hinkley (1997).
\begin{algorithm}
        \caption{Linear Regression Cases Resampling}\label{cases resampling}
        \begin{algorithmic}[1]
            \For {$b=1,\ldots,B$ }
            \State Draw a bootstrap sample $Z_b^*$ of size $n$ from the cases $(X_i,Y_i)$ to obtain $Z_b^*=\{(X_1^*,Y_1^*),\ldots,(X_n^*,Y_n^*)\}$.
            \State Fit least squares regression to $(X_1^*,Y_1^*),\ldots,(X_n^*,Y_n^*)$ to obtain the estimates $\hat{\beta}_b^*$ and $\hat{s}_b^{*2}$.
            \EndFor
        \end{algorithmic}
    \end{algorithm}
Another approach we could take is to resample the residuals of the
linear model. To motivate this approach, note that the vector of raw
residuals \((e_1,\ldots,e_n)\) approximate the error term
\(\varepsilon\). That is, assuming the bivariate distribution of
\((X,Y)\) is such that the specification of the linear model is correct,
then given the true value of \(\beta\), the error terms would be given
by \(\varepsilon_i=Y_i-X_i\beta\). As \(\beta\) is generally unknown,
the best we can do is to form estimates of the error terms
\(\hat{\varepsilon}_i=Y_i-\hat{Y_i}=Y_i-X_i\hat{\beta}\). We now have a
vector of estimated error terms
\(\hat{\varepsilon}=(\hat{\varepsilon}_1,\ldots,\hat{\varepsilon}_n)\)
which we could consider as being an i.i.d. sample if the cases
\((X_i,Y_i)\) were independently sampled from the distribution of
\((X,Y)\). As Efron and Tibshirani note, we could consider obtaining any
particular component of the vector \(\hat{\varepsilon}\) as being from
the EDF \(\hat{F}\) of the true distribution of the error terms \(F\),
where we place a probability mass of \(\frac{1}{n}\) to drawing any one
of the \(\hat{\varepsilon}_i\). Taking this conceptual approach, we
could then draw bootstrap samples of the residuals
\(\hat{\varepsilon}_1,\ldots,\hat{\varepsilon}_n\). Consider the \(X_i\)
component of the cases \((X_i,Y_i)\) fixed and the estimate
\(\hat{\beta}\) as fixed. Resample with replacement the vector
\(\hat{\varepsilon}\) to obtain the bootstrapped vector of residuals,
\(\hat{\varepsilon}^*=(\hat{\varepsilon}_1^*,\ldots, \hat{\varepsilon}_n^*)\).
Then for each \(i=1,\ldots,n\) our bootstrapped response variable would
be \(Y_i^*=X_i\hat{\beta}+\hat{\varepsilon}_i^*\). Denoting
\(X_i^*=X_i\), we obtain the bootstrapped dataset
\(Z^*=\{(X_1^*,Y_1*),\ldots,(X_n^*,Y_n^*)\}\). We then run a linear
regression again on \(Z^*\) to obtain estimates \(\hat{\beta}^*\) and
\(\hat{s}^{*2}\). We present what was just discussed in the following
algorithm.
\begin{algorithm}
        \caption{Bootstrapping Residuals of Linear Regression Model}\label{residuals resample}
        \begin{algorithmic}[1]
            \State Fit a linear regression model to the data $Z=\{(X_1,Y_1),\ldots,(X_n,Y_n)\}$ to obtain the estimate $\hat{\beta}$. 
            \State Compute the estimated residuals vector $\hat{\varepsilon}=(\hat{\varepsilon}_1,\ldots,\hat{\varepsilon}_n)$, where $\hat{\varepsilon}_i=Y_i-X_i\hat{\beta}$.
            \For {$b=1,\ldots, B$ }
                \State Sample with replacement $n$ times from $\hat{\varepsilon}$ to obtain the bootstrapped residuals $\hat{\varepsilon}^{*b}=(\hat{\varepsilon}_1^{*b},\ldots,\hat{\varepsilon}_n^{*b})$.
                \For {$i=1,\ldots, n$ }
                    \State Set $X_i^{*b}=X_i$ and $Y_i^{*b}=X_i\hat{\beta}+\hat{\varepsilon}_i^{*b}$
                \EndFor
                \State Obtain the bootstrapped dataset $Z^{*b}=\{(X_1^{*b}, Y_1^{*b}),\ldots, (X_n^{*b},Y_n^{*b})\}$.
                \State Fit a linear regression model on the bootstrap replicate of the data $Z^{*b}$ to obtain the estimates $\hat{\beta}^{*b}$ and $(\hat{s}^2)^{*b}$.
            \EndFor
        \end{algorithmic}
    \end{algorithm}
Note that the two methods of resampling the linear regression model rely
on different assumptions on the data. In particular, we are assuming
that the error structure of the data does not rely on our predictors
\(X_i\). By fixing the \(X_i\), we are assuming the distribution of the
error terms \(F\) is invariant to the data we have observed. This is a
fairly strong assumption. Furthermore, resampling the residuals works
well only if the variance of the errors is homoskedastic. If the
variance of the errors is heteroskedastic, then the modeling assumptions
we made are invalid and a weighted linear regression approach or wild
bootstrap approach may be necessary. The cases resampling is much less
sensitive to the modeling assumptions we make as we do not generate the
\(Y_i^*\)'s in the same way as with the residuals method. If either of
these two methods are applicable, we could then proceed to form summary
statistics such as the sample mean or sample standard deviation of the
coefficients and the estimate of the variance of the model. The two
methods we presented for bootstrapping the linear regression are easily
adapted to other types of models. In particular, we could consider
applying these methods to highly non-linear estimators, which is
precisely what we do with bagged and random forests with respect to CART
trees.

\section{Random Forests}\label{random-forests}

Among the limitations of CART discussed in the preceding sections, the
biggest issues is perhaps the variability of the method and the issue of
collinearity among predictors. While overfitting can be addressed using
cost-complexity pruning, variability and collinearity are not fixed by
pruning. Random forests deal with these two issues of CART by
introducing a resampling and randomization mechanism. The natural order
is to first discuss bagged forests before turning to random forests.
\begin{algorithm}
        \caption{Bagged Forest algorithm}\label{bagged forest}
        \begin{algorithmic}[1]
            \For {$b=1,\ldots, B$ }
            \State Draw a bootstrap sample $Z_b^*$ of size $N$ from the training data $Z$.
            \State Grow a CART tree $T_b$ on each bootstrap sample $Z_b^*$.
            \EndFor
            \State Output the bagged forest ensemble $\{T_b\}_{b=1}^B$.
        \end{algorithmic}
    \end{algorithm}
One method of improving CART trees is to bag them. Bagging, which stand
for bootstrap aggregating, is a variation reduction technique
particularly useful for improving the predictive power of weak learners.
We are interested in bagging CART trees to reduce the variability of
single trees under slight perturbations of the data. Generally bagged
ensembles of learners produce robust predictions in comparison to
running the learner once.

\subsection{How Bagged Forests Work}\label{how-bagged-forests-work}

In particular, bagged forests are an ensemble obtained by taking many
bootstrap samples of the data and fitting a tree to each bootstrapped
dataset. In this case, we grow the trees quite deep and do not employ
cost-complexity pruning. The idea behind this decision is that we want
to sufficiently explore the feature space using the tree ensemble, and
since we are also taking an average of the trees, we are fine with
overfitting at least a little bit. As the algorithm above for bagged
forests indicates, the output is an ensemble of trees
\(\{T_b\}_{b=1}^B\). Given a test data point, we form a prediction by
taking the average of predictions given by the tree ensemble:
\[\hat{f}_{bf}^B(x)=\frac{1}{B}\sum_{b=1}^B T_b(x),\] where
\(\hat{f}_{bf}^B\) indicates we are taking the bagged forest estimate of
the underlying regression or classification function using \(B\) many
trees. Note that it is important to be consistent in the choice of
splitting criterion throughout the bagging process.

It has been shown by Friedman and Hall (2000) and Chen and Hall (2003)
that bagging is especially effective when used on highly non-linear
models such as CART trees. Under ideal conditions, bagged estimates of
non-linear estimators reduces both the bias and variance of the
estimator. So bagging trees seem to be effective because of the
reduction to the variance of the estimator, which produces a more robust
prediction.

When we bag trees each observation in the data is not used within each
individual tree. A bootstrap replicate \(Z_b\) of the data \(Z\) will
likely exclude some of the observations within the data. The
observations used within the bootstrap replicate \(Z_b\) is called the
in-bag data while the observations not used within the bootstrap
replicate \(Z_b\) is called the out-of-bag (OOB) data and is denoted by
\(\bar{Z}_b\). The OOB data allows us to approximate the test error of
the ensemble as follows. For simplicity suppose we are in the regression
setting (the classification setting is similar). Then the OOB estimate
of the MSE of the bagged forest is given by
\[MSE_{\textup{OOB}}(T;Z)=\frac{1}{B}\sum_{b=1}^B MSE(T_b;\bar{Z}_b).\]
As the number of trees in the ensemble grows, the OOB estimate of the
MSE for the bagged forest converges to the LOOCV estimate of the MSE for
forest (Friedman et al.).

A weakness of bagged forests is collinearity between trees (Friedman et
al.). This collinearity between trees grown from the bootstrap sample
can be addressed by adding a randomization mechanism in the tree growing
process. When we grow a tree from a bootstrap sample, even with the
randomness induced by resampling from the data, certain features may be
explored at the expense of other just as interesting features. This is a
particular issue with collinear predictors. If two predictors are
collinear, then the bagged forest might consistently choose one
predictor over another even if the predictor not chosen leads to splits
that are just as informative. This is due in part to the greedy nature
of the CART algorithm when searching for optimal splits over the feature
space (Cite ESL or some similar textbook for this point). The algorithm
does not take into account the second best or third best splits.
Furthermore, it is not difficult to see that bagging will generally
produce an ensemble of trees that are quite similar to one another,
subject to some perturbations. These trees will be strongly correlated
with other trees in the ensemble, so if there is a less explored part of
the feature space, then the ensemble will struggle to produce good
predictions over that part of the feature space.

\subsection{The Random Forest
Algorithm}\label{the-random-forest-algorithm}

Random Forests try to deal with this issue of correlated trees and
collinearity among predictors by choosing at random only \(m\leq p\) of
the predictors to be considered as candidate splitting variables at each
split in each tree in the ensemble. This randomness further reduces the
variance of the bagged forest by decorrelating the trees in the
ensemble. Furthermore, while individually the trees may perform worse
than a single pruned tree, collectively the ensemble has a better chance
of exploring the feature space fully.
\begin{algorithm}
        \caption{Random Forest algorithm}\label{random forest}
        \begin{algorithmic}[1]
            \For {$b=1,\ldots, B$ }
            \State Draw a bootstrap sample $Z_b^*$ of size $N$ from the training data $Z$.
            \State Grow the CART tree $T_b$ on $Z_b^*$ with the following modification:
            \While {minimum node size $n_{\min}$ not reached across $T_b$} 
            \State Select $m\leq p$ candidate splitting variables at random.
            \State Pick the best splitting variable and splitting point among the $m$ variables selected at random.
            \State Split the node into two daughter nodes.
            \EndWhile
            \EndFor
            \State Output the random forest ensemble $\{T_b\}_{b=1}^B$.
        \end{algorithmic}
    \end{algorithm}
To form a prediction at a test point \(x\), we have in the regression
setting \[\hat{f}_{rf}^B(x)=\frac{1}{B}\sum_{b=1}^B T_b(x).\] {[}Add in
the classification setting{]}.

\subsection{Variable Importance
Measures}\label{variable-importance-measures}

One of the advertised outputs of random forests is the variable
importance (VI) measure. There are two main variable importance measures
in common use, with the choice of VI measure varying depending on the
setting and splitting criterion chosen. The first choice is the Mean
Decrease in Impurity (MDI) which is typically used in the classification
setting where the GINI index or Shannon entropy is used as the splitting
criterion. The second choice is Mean Decrease in Accuracy (MDA) which is
typically used in the regression setting where RSS has been used as the
splitting criterion. The idea of MDI is to find how much the nodal
impurity \(p(t)\Delta i(s,t)\) decreases for all nodes \(t\) in which
the variable of interest \(X_j\) is used and to take that average over
all trees in the ensemble. More important variables are those which are
on average more often chose for splits and which also contribute most to
reducing the nodal impurity of the trees.
\begin{algorithm}
        \caption{MDI Variable Importance}\label{mdi variable importance}
        \begin{algorithmic}[1]
            \State Grow a random forest $\{T_b\}_{b=1}^B$.
            \For {$j=1,\ldots,p$ }
                \For {$b=1,\ldots,B$ }
                \State Compute the importance of $X_j$ in $T_b$ as $VI_b(X_j)=\sum_{t\in T_b} \mathbb{I}(j_t=j)p(t)\Delta i(s,t)$ to be the sum of the decrease in impurity over nodes where variable $X_j$ is used .
                \EndFor
                \State Compute the importance of $X_j$ in the random forest to be $VI(X_j)=\frac{1}{B}\sum_{b=1}^B VI_b(X_j).$
            \EndFor
        \end{algorithmic}
    \end{algorithm}
The idea of MDA is to measure for each variable \(X_j\), on average how
much the predictive accuracy of the forest as measured using RSS suffers
when the \(X_j\) component is permuted across observations within the
OOB dataset \(\bar{Z}_b\) of each tree \(T_b\) in the ensemble.
\begin{algorithm}
        \caption{MDA Variable Importance}\label{mda variable importance}
        \begin{algorithmic}[1]
            \State Grow a random forest $\{T_b\}_{b=1}^B$.
            \For {$j=1,\ldots,p$ }
            \For {$b=1,\ldots,B$ }
            \State Permute the $X_j$ component of $Z_b$ to obtain the dataset $Z_b^j$, where $X_j$ has been permuted.
            \State Compute the importance of $X_j$ in $T_b$ to be $VI_b(X_j)=\frac{1}{|\bar{Z}_b|}(RSS(T_b,Z_b)-RSS(T_b,Z_b^j)).$
            \EndFor
            \State Compute the importance of $X_j$ in the random forest to be $VI(X_j)=\frac{1}{B}\sum_{b=1}^B VI_b(X_j).$
            \EndFor
        \end{algorithmic}
    \end{algorithm}
More important variables in the random forests are those for which the
variable importance is large, as those are the variables for which the
predictive accuracy of the random forest suffers the most. Note that as
variable importance measures currently defined and used, the threshold
for importance of a variable is something which the researcher has to
decide.

\subsection{Issues with Random
Forests}\label{issues-with-random-forests}

While random forests are good out of the box predictors, there are
situations where the random forest algorithm can fail to produce good
predictions. If the underlying regression function the random forest is
linear, if the predictors are highly correlated, or if the data cannot
be bootstrapped, then the random forest will not perform well. Random
forests are especially adept at handling highly non-linear functions,
but can struggle with linear response compared to say linear regression.

If the predictors are highly correlated, then ithas been shown by Strobl
et al. that the trees within the forest ensemble will be biased and the
variable importance measures will not be reliable due to confounding
between similar looking variables. While the randomization step in
random forest algorithm can alleviate the correlation between trees in
the ensemble, the collinearity between predictors can cause the
predictive performance of the random forest to suffer. This is an issue
with the underlying CART algorithm and Hothorn, Strobl, and their
collaborators developed Conditional Inference trees and forests to deal
with the issue of collinearity in the data (which will be explained in
the next chapter). While using random forest variable importance for
variable selection certainly seems like a good idea, the issue of
collinear data makes the applicability of variable importance measures
less immediate. Perhaps a proper use of random forest variable
importance measures as a variable selection procedure would be to run a
random forest only after some dimension reduction techniques have been
applied. If the collinearity of the predictors has been dealt with, then
heuristically the variable importance measures should be able to more
accurately estimate the importance of predictors within the ensemble.

Finally, there are situations where the data cannot be bootstrapped.
This could be due to a number of factors including that the data has a
heavily-tailed distribution or if there are particularly extreme values.
In this case, a different bootstrap scheme could perhaps remedy the
situation where the naive bootstrap fails. One common important
resampling scheme used within random forests is to use the m-out-of-n
bootstrap (also referred to subsampling or subagging in the literature).
The m-out-of-n bootstrap is a resampling scheme which resamples with or
without replacement \(m\leq n\) observations from the data to form
datasets with \(m\) many data points to run the bootstrap computation.
Bickel et al. have shown that in important cases, subsampling can
succeed where the bootstrap fails. Of course, using less of the
available data is less efficient, but in the context of random forests
this loss of efficiency may not be an issue. In {[}Cite paper comparing
CI trees to CART trees using IJ{]}, simulation results showed that
subsampling instead of using the standard nonparametric bootstrap can
improve the performance of random forests. The authors suggested that
subsampling further reduces the variance of the ensemble by producing
trees that are even more decorrelated. Their heuristic is that there is
a lower probability of duplicate data points being chosen using
subsampling (this probability is zero if we are subsampling without
replacement), furthermore there is a lower probability of highly
correlated data points being chosen as one of the \(m\leq n\) points.
This certainly seems plausible, but we would like to see further
simulation results or a technical result that explains why subsampling
works well for forests. We would also like to note that most consistency
results (Biau et al., Scornet et al., Ishwaran et al., and Wager et al.)
make this subsampling assumption in their analysis of the random forest
model, as subsampling constructions make the forest ensemble more
amenable to mathematical analysis.

\section{Focus of this Thesis}\label{focus-of-this-thesis}

In the previous three sections we discussed the basics of trees, the
bootstrap, and random forests. We would like to now discuss the
direction we are going to take this thesis. We are interested in
developing inferential tools using the random forest algorithm. The
random forest often produces good predictions out of the box with little
tuning required except growing a sufficient number of trees and choosing
a good value of \(m\leq p\) to try at each split in the tree growing
process. The ease of fitting random forests is one of the advantages of
the random forest compared to more complex methods like neural networks
or support vector machines. Even if the random forest is easier to fit
than neural nets or SVMs, the underlying random forest mechanism is
quite complex. This complexity makes developing inferential tools with
random forests difficult. In the next chapter we will discuss some
approaches that have been developed recently by researchers. Of
particular interest for us are the variable importance measures for
random forests. There is the MDI variable importance and the MDA
variable importance measure. While there are concerns about the bias in
the MDI and MDA VI measures with random forests using CART tree's due to
how CART tree's are constructed (Strobl et al., 2008), we would like to
extend the inferential capabilities of random forests using CART as this
is perhaps the most popular version of the algorithm in use.

The random forest VI can be viewed as a random variable with an unknown
probability distribution. While determining the distribution of random
forest VI's would be ideal towards developing inference for random
forests, finding the distribution of random forest VI's is difficult and
unclear in general. However, note that the bootstrap provides a
computational method of approximating the sampling distribution of
random forest VI measures. We could then proceed to obtaining estimates
of the standard error and to form approximate interval estimates. We
could grow a random forest once, obtain an estimate of the VI of each
variable \(X_j\) and then run a bootstrap to obtain estimates of the
standard error of each variable. Inference could then proceed using the
VI measures along with the estimate of standard error. Note that in this
scheme there is two levels of bootstrap at play. There is the bootstrap
within each random forest and then the bootstrapping to obtain the
standard error estimate. The estimate we produce of the VI of a variable
\(X_j\) can be erratic. Running the random forest again can produce a
new estimate of the VI of variable \(X_j\). As Efron suggests in his
2014 article ``Estimation and Accuracy after Model Selection,'' we could
remedy the erratic, non-smooth nature of the VI estimate by bagging
\(VI(X_j)\). However, when we bag \(VI(X_j)\) we would require a
third-level of bootstrapping to produce an estimate of the standard
error of the bagged estimate of \(VI(X_j)\). This is computationally
expensive, especially considering that the computational cost of the
random forest depends on the number of observations, number of tree's
grown, and number of variables present. However, applying Efron's
infinitesimal jackknife (IJ) technique, which we will discuss in the
next chapter, we can efficiently estimate the standard error of the
bagged estimate of \(VI(X_j)\) only using the two levels of
bootstrapping. In this thesis we will be exploring the properties of IJ
estimates of the standard error of the bagged estimate \(VI(X_j)\)
utilizing different bootstrap schemes. Utilizing the IJ, we can then
produce interval estimates and confidence intervals of the variable
importance of variables in the random forest. This allows us to
characterize the uncertainty of \(VI(X_j)\) when using variable
importance measures for description, variable selection, and inference.

\section{Outline of Remaining
Chapters}\label{outline-of-remaining-chapters}

Chapter 2: Inference and variable importance for random forests. In this
chapter we will discuss the various approaches to inference for random
forests that have been developed by researchers. These include Ishwaran
and Louppe's analysis of VI measures of RF. Ishwaran's analysis is of
the variable importance scheme he devised. Ishwaran's scheme is amenable
to theoretical analysis and he produces asymptotic results for his
scheme. Louppe analyzes the MDI variable importance in the context of a
construction called totally randomized forests in the setting of finite
sample spaces. Louppe investigates the effect of relevant and irrelevant
variables to MDI VI in this setting and provides asymptotic results.
These are two of themain papers investigating properties of VI.

The other schemes include Mentch and Hooker's prediction intervals which
are produced by interpreting the random forest ensemble as a particular
type of U-statistic and applying the U-statistics theory to produce
prediction intervals. They then can produce hypothesis tests and
confidence intervals in this context.

A second approach towards prediction intervals is Wager's application of
the IJ to the random forest estimator to produce prediction intervals
and confidence intervals. Wager has a couple of papers in this area
which we will discuss. In this part we will also go into greater detail
about the IJ, it's properties, and why it works.

We will also discuss the conditional inference trees and conditional
variable importance developed by Strobl, Hothorn, and their colleagues
to deal with the biased manner in which the trees in random forests are
constructed. In this part we will also discuss the Infforest variable
importance measure developed by Aurora.

Chapter 3: In this chapter we will go into great detail about the
bootstrap and why it works. We will discuss the technical details of
different methods of estimating the standard error of functional
statistics, and we will go into the construction of different bootstrap
confidence intervals along with properties of these confidence
intervals. We will also discuss different bootstrap schemes and in
particular subsampling, and why subsampling works well, especially in
the random forest context.

Chapter 4: We will present our different approaches to bootstrapping
variable importance measures and provide simulation results. The first
approach is the cases bootstrap, the second approach is the
bootstrapping residuals of the random forest approach. Within the cases
approach, there are two ways we will try to constrain the variability of
bootstrap replicates. Adopting the notation from general linear models,
we could fit a pruned tree predicting \(Y\sim X_1,\ldots,X_p\). Then we
would bootstrap from the terminal nodes of this single tree to try to
produce bootstrap replicates which capture the relationship between the
variables. Another approach is for each predictor \(X_j\), fit a single
well pruned tree \(X_j\sim X_1,\ldots,X_{j-1},X_{j+1},\ldots,X_p\). We
would then draw bootstrap samples from the terminal nodes of this tree
which respect the partition produced by these terminal nodes. The idea
is to fix the relationship between every predictor not \(X_j\) and allow
the \(X_j\)'s to be exchanged across the bootstrap samples. We would
produce simulation results of each approach and compare the statistical
properties of these different constructions.

Chapter 5: If all goes well there might be a chapter 5 where we explore
large sample properties and distribution of VI measures. This is likely
going to be a difficult mathematical problem to solve in general, so a
tangible course of action might be to pick a generating mechanism and
easy VI measure from which we could figure out the VI distribution under
some set of strong assumptions.

\chapter{Variable Importance and Inference for Random
Forests}\label{variable-importance-and-inference-for-random-forests}

\section{Introduction}\label{introduction-1}

In this chapter we focus on approaches of inference for random forests
involving variable importance measures of random forests. Such
approaches involve using variable importance measures of random forests
to evaluate the relative importance of different variables in the
construction of the forest. Louppe (2014) and Ishwaran (2007) focused on
theoretical properties of variable importance measures while Owens
(2017), Strobl, Boulesteix, Zeileis, \& Hothorn (2007), and Strobl,
Boulesteix, Kneib, Augustin, \& Zeileis (2008) are centered on
developing less biased variable importance measures for hypothesis
testing.

\section{Theoretical Analysis of MDA Variable Importance
Measures}\label{theoretical-analysis-of-mda-variable-importance-measures}

The approach to variable importances adopted in Ishwaran (2007) differs
from the original variable importance measures for random forests, so we
require some additional vocabulary.

Suppose \(T\) is a binary recursive tree and suppose that \(T\) has
\(M\) many terminal nodes. For each point \(\mathbf{x}\) in the feature
space, \(T\) maps \(\mathbf{x}\) to one of the \(M\)-many terminal
nodes. In particular, if we let \(\mathcal{X}\) denote the feature
space, then \(T\) is a function
\(\mathcal{X}\rightarrow \{1,\ldots, M\}\) defined by the equation
\[T(\mathbf{x})=\sum_{m=1}^M m B_m(\mathbf{x}),\] where
\(B_m(\mathbf{x})\) is a \(0-1\) basis function which partition the
feature space \(\mathcal{X}\).

Let \(Z=\{(\mathbf{x}_i,Y_i)|i=1,\ldots,n\}\) denote the training data,
where \(\mathbf{x}_i\) is a covariate in the feature space and \(Y_i\)
is the response. We call \(T\) a binary regression tree if it is a
binary recursive tree grown from \(Z\) using using binary recursive
splits of the form \(x_j\leq c\) and \(x_j> c\) where split values \(c\)
are chosen based on the observed \(\mathbf{x}_i\) in the training data
\(Z\). The value \(a_m\) in the terminal node is the average response of
the training observations falling in the \(m\)th node. That is,
\[a_m=\frac{\sum_{i=1}^n \mathbb{I}\{T(\mathbf{x}_i)=m\} Y_i}{\sum_{i=1}^n \mathbb{I}\{T(\mathbf{x}_i)=m\}}.\]
Note that \(\mathbf{x_i}\) denotes a row of covariates for the training
data \(Z\), while \(x_j\) denotes the \(j\)th variable along the columns
of the training data.

For a binary regression tree, the basis functions \(B_m(\mathbf{x})\)
are product splines of the following form:
\[B_m(\mathbf{x})=\prod_{l = 1}^{L_m} [x_{l(m)}-c_{l, m}]_{s_{l,m}},\]
where \(L_m\) denotes the number of splits used to construct
\(B_m(\mathbf{x})\). For each split \(l\), there is a splitting variable
\(\mathbf{x}_{l(m)}\) which denotes the \(l(m)\)th coordinate of
\(\mathbf{x}\) and a splitting value \(c_{l,m}\). The \(s_{l,m}\) are
binary \(\pm 1\) values, where for a given scalar \(x\),
\([x]_{+1}=\mathbb{I}(x>0)\) and \([x]_{-1}=\mathbb{I}(x\leq 0).\) Note
that the basis functions satisfy an orthogonality property, which gives
\(B_m(\mathbf{x})B_{m'}(\mathbf{x})=0\) if \(m\neq m'\). Note also that
given a tree \(T\), the predictor associated with the tree can be
written as a linear combination of basis functions:
\[\hat{\mu}(\mathbf{x})=\sum_{m=1}^M a_m B_m(\mathbf{x}).\] We are now
prepared to define Ishwaran's variable importance measure.

Informally, the \(MDA\) variable importance of a variable \(x_j\) is the
difference between \(MSE\) of the tree \(T\) when \(x_j\) is randomly
permuted and \(MSE\) of the tree \(T\) when \(x_j\) is not permuted. As
such a scheme of variable importance is difficult to analyze, Ishwaran
proposes a surrogate measure. For the variable \(x_j\), we drop
\(\mathbf{x}\) down the tree and follow the binary splits until either a
terminal node is reached or a node with a split depending on \(x_j\) is
reached. If a node with a split depending on \(x_j\) is reached, we then
subsequently assign \(\mathbf{x}\) randomly to either the left or right
daughter node, whenever there is a split, until we reach a terminal
node. The difference in \(MSE\) between noising up \(x_j\) and not
noising up \(x_j\) to be the variable importance of \(x_j\) in the tree
\(T\). Denote the tree that results from noising up \(x_j\) by \(T_j\).

Such a scheme relies on the following heuristic: if we chose an adequete
splitting rule to construct our tree, then we expect that variables that
are split earlier in the tree are more important, since prediction will
suffer the most from noising up a variable higher up in the tree than a
variable close to a terminal node. This is a behavior observed in CART
trees and random forests based on CART: splits closer to the root node
are more influential than splits close to terminal nodes, so the MDA or
MDI variable importance of variables split on close to the root node are
expected to be higher than otherwise.

\subsection{Maximal Subtrees and Theoretical
Results}\label{maximal-subtrees-and-theoretical-results}

Defining a structure on binary regression trees called subtrees, we can
write the predictor for the noised up tree as a deterministic component
relying on terminal nodes for no parent nodes involve a split on
\(x_j\), and a random component involving terminal nodes for which there
are parent nodes involving a split on \(x_j\). The definition of the
subtree is quite intuitive. We call \(\tilde{T}_j\) a \(j\)-subtree of
the tree \(T\), if the root node of \(\tilde{T}_j\) has daughters that
depend on an \(x_j\) split. A \(j\)-subtree \(\tilde{T}_j\) is a maximal
\(j\)-subtree of the tree \(T\), if there are no larger \(j\)-subtrees
continaing \(\tilde{T}_j\). For a given tree \(T\) and for each variable
\(x_j\), there is a set of \(K_j\) many distinct maximal \(j\)-subtrees,
which are denoted by \(\tilde{T}_{1,j},\ldots, \tilde{T}_{K_j,j}\). Note
each distinct \(\tilde{T}_{k,j}\) maximal \(j\)-subtree contains a set
of distinct terminal nodes \(M_{k,j}\). Each \(M_{k,j}\) is distinct for
\(k=1,\ldots,K_j\), since we are working with maximal \(j\)-subtrees.
Define \[M_j=\bigcup_{k=1}^{K_j} M_{k,j}\] to be the set of terminal
nodes for which there is a parent node involving a split on \(x_j\).
Ishwaran (2007) proves the following lemma about the functional form of
the predictor for the tree \(T_j\).
\begin{lemma}
Let $\hat{\mu}_j(\mathbf{x})$ denote the predictor for $T_j$. Then $$\hat{\mu}_j(\mathbf{x})=\sum_{m\notin M_j}a_m B_m(\mathbf{x})+\sum_{k=1}^{K_j} \tilde{a}_{k,j}\mathbb{I}\{T(\mathbf{x})\in M_{k,j}\},$$ where $\tilde{a}_{k,j}$ is the random terminal value assigned by $\tilde{T}_{k,j}$ under the random left right path through $\tilde{T}_{k,j}$. We write $\tilde{P}_{k,j}$ to denote the distribution of $\tilde{a}_{k,j}$. 
\end{lemma}
For a proof, see Ishwaran (2007). It is a bit surprising that the
functional form of \(\hat{mu}_j(\mathbf{x})\) can be separated into the
two components. Given this lemma and the definition of \(j\)-subtrees,
we can more formally define the variable importance of \(x_j\) in the
tree \(T\).

Let \(g\) be a loss function. Often we use the squared error to evaluate
loss, which corresponds to \(MSE\), but there is no strict requirement
in the definition. Denote the test data by \((Y,\mathbf{x})\). Then the
prediction error of the predictor \(\hat{\mu}\) is given by
\(\mathbb{E}(g(Y,\hat{\mu}(\mathbf{x})))\). As a reminder, we assume
that there is an underlying regression function
\[Y=\mu(\mathbf{x})+\varepsilon,\] where \(\varepsilon\) is independent
error with zero mean and variance \(\sigma^2>0\). Similarly, we can
define the prediction error of the predictor \(\hat{\mu}_j\) to be
\(\mathbb{E}(g(Y,\hat{\mu}_j(\mathbf{x}))).\) Set
\(g(Y,\hat{\mu}(\mathbf{x}))=(Y-\hat{\mu}(\mathbf{x}))^2\) to be the
\(L_2\) loss, which corresponds to \(MSE\). Define the variable
importance of the variable \(x_j\) to be
\[\Delta_j=\mathbb{E}((Y-\hat{\mu}_j(\mathbf{x}))^2)-\mathbb{E}((Y-\hat{\mu}(\mathbf{x}))^2).\]
Application of the lemma and some manipulation allows us to write
\[\Delta_j=\mathbb{E}(R_j(\mathbf{x})^2)-2\mathbb{E}\left(R_j(\mathbf{x})[\mu(\mathbf{x})-\hat{\mu}(\mathbf{x})]\right),\]
where
\[R_j(\mathbf{x})=\sum_{k=1}^{K_j}\sum_{m\in M_{k,j}} (\tilde{a}_{k,j} -a_m)B_m(\mathbf{x}).\]

As noted in Ishwaran (2007), we make the assumption that the true
regression function \(\mu\) is of similar form to \(T\). That is, assume
\[\mu(\mathbf{x})=\sum_{m=1}^M a_{m,0} B_m(\mathbf{x}),\] where
\(a_{m,0}\) are the true, but unknown, terminal values. Under this and
some other large sample assumptions, Ishwaran finds that asymptotically,
each maximal \(j\)-subtree will tend to contribute equally to the
variable importance \(\Delta_j\). In effect, nodes closer to the root of
a maximal \(j\)-subtree will have a larger effect on \(\Delta_j\) than
nodes closer to the terminal nodes.

\subsection{Extension to Forest
Ensembles}\label{extension-to-forest-ensembles}

The framework developed in Ishwaran (2007) extends naturally to forest
ensembles and his theoretical result regarding forest ensembles provides
some information of the behavior of variable importance measures for
random forests. First for some notation, recall that in the forest
ensemble setting, we draw \(B\) many bootstrap resamples of the training
data to obtain the bootstrap replicates
\(Z^{b}=\{(\mathbf{x}_i^b, Y_i^b)|i=1,\ldots,n\}\) of the training data
for \(b=1,\ldots, B\). We then construct a binary regression tree
\(T(\mathbf{x};b)\) on each bootstrap replicate of the data and have the
forest \(\hat{\mu}_F\) as the average of predictions over the trees
\(T(\mathbf{x};b)\):
\[\hat{\mu}_F(\mathbf{x})=\frac{1}{B}\sum_{b=1}^B \hat{\mu}(\mathbf{x};b),\]
where \(\hat{\mu}(\mathbf{x};b)\) denotes the predictor for the tree
\(T(\mathbf{x};b)\). Given that each \(\hat{\mu}(\mathbf{x};b)\) is a
linear combination of basis functions, we can write
\[\hat{\mu_F}(\mathbf{x})=\frac{1}{B}\sum_{b=1}^B\sum_{m=1}^{M^b}a_m^b B_m(\mathbf{x};b).\]
Again assume that the \(\mu(\mathbf{x})\) has a similar structure to
\(\mu(\mathbf{x})\). That is,
\(\mu(\mathbf{x})=\frac{1}{B}\sum_{b=1}^B\sum_{m=1}^{M^b}a_{m,0}^b B_m(\mathbf{x};b)\),
where \(a_{m,0}\) are the true, but unknown, terminal values. We denote
the noised up forest predictor for the variable \(x_j\) by
\(\hat{\mu}_{F_j}(\mathbf{x})\). As with the normal forest predictor,
the noised up forest predictor is the average of the noised up
predictors \(\hat{\mu}_j\) over the bootstrap resamples:
\[\hat{\mu}_j(\mathbf{x})=\sum_{b=1}^B \hat{\mu}_j(\mathbf{x};b).\] As
forest predictors are simply the average of individual trees, we can
extend the lemma to \(\hat{\mu}_{F_j}\) with the use of \(b\)'s in
appropriate places denoting the usage of the \(b\)th bootstrap resample.
That is, we can write
\[\hat{\mu}_{F_j}(\mathbf{x})=\frac{1}{B}\sum_{b=1}^B\left(\sum_{m\notin M_j^b} a_m^b B_m(\mathbf{x};b)+\sum_{k=1}^{K_j^b} \tilde{a}_{k,j}^b \mathbb{I}\{T(\mathbf{x};b)\in M_{k,j}^b\} \right).\]
The variable importance of the variable \(x_j\) is defined to be
\[\Delta_{F_j}=\mathbb{E}((Y-\hat{\mu}_{F_j}(\mathbf{x}))^2)-\mathbb{E}((Y-\hat{\mu}(\mathbf{x}))^2).\]
Similar to the single tree case, the variable importance of the variable
\(x_j\) in the forest ensemble can be written as
\[\Delta_{F_j}=\mathbb{E}(R_{F_j}(\mathbf{x})^2)-2\mathbb{E}\left(R_{F_j}(\mathbf{x})[\mu(\mathbf{x})-\hat{\mu}_F(\mathbf{x})]\right),\]
where
\[R_{F_j}(\mathbf{x})=\frac{1}{B}\sum_{b=1}^b\sum_{k=1}^{K_j^b} \sum_{m\in M_{k,j}^b} (\tilde{a}_{k,j}^b-a_m^b)B_m(\mathbf{x}; b).\]

We are now ready to state a result about the asymptotic form of
\(\Delta_{f,j}\):
\begin{theorem}
Let $R_{F_j, 0}(\mathbf{x})$ be the function $R_{F_j}(\mathbf{x})$, in which each instance of $a_m^b$ has been replaced with $a_{m,0}^b$. Assume that $Y$ can be written in terms of a regression model $$Y=\mu(\mathbf{x})+\varepsilon$$ and that $$\mu(\mathbf{x})=\frac{1}{B}\sum_{b=1}^B\sum_{m=1}^{M^b}a_{m,0}^b B_m(\mathbf{x};b).$$ If $a_m^b\rightarrow a_{m,0}^b$ for each $m$ and $b$, then $$\Delta_{F_j}\rightarrow \mathbb{E}\left(R_{F_j,0}(\mathbf{x})^2\right)\leq\mathbb{E}\left(\frac{1}{B}\sum_{b=1}^b\sum_{k=1}^{K_j^b}\theta_0(k,j,b)\right),$$ where $$\theta_0(k,j,b)=\sum_{m\in M_{k,j}^b} \pi_{m,b}\tilde{P}_{k,j,0}^b\left(\tilde{a}_{k,j,0}^b-a_{m,0}^b\right)^2$$ is the node mean squared error for the $k$th maximal $v$-subtree of $T(\mathbf{x};b)$and $\pi_{m,b}=\mathbb{P}(B_m(\mathbf{x};b)=1|Z)$. 
\end{theorem}
For a proof, see Ishwaran (2007). As Ishwaran (2007) notes, the bound in
the above theorem becomes tighter as the trees in the forest become more
and more orthogonal to each other. The above theorem is applicable if
the forest predictor is consistent. This suggests that forest ensemble
methods which are consistent and produce trees that are at least
approximately orthogonal to each other allow for variable importance to
be characterized through node mean squared error of subtrees. Variable
importance as defined above differs from the \(MDA\) variable importance
for random forests, but the two follow a similar heuristic towards
measuring variable importance via the loss of predictive accuracy when
the variable \(x_j\) is pertrubed. Therefore this result suggests that,
perhaps, so long as the random forest estimator is consistent and the
trees constructed are at least approximately orthogonal, the \(MDA\)
variable importance can be characterized via the mean squared error of
some analogous structure to maximal subtrees. This is, of course,
conjectural, but we are interested whether for simpler forest
algorithms, results similar to the theorem can be proven. Asymptotic
results for the \(MDA\) and also \(MDI\) variable importance measures
are difficult to formulate and prove for several reasons. First, the
\(CART\) tree construction process combined with the bootstrapping and
randomization procedure of the random forest is difficult to
mathematically analyze. Second, it is still unknown whether random
forests constructed via \(CART\) are consistent. However, there are
other random forest variants known to be consistent, so those variants
may be a natural starting point to try to prove asymptotic results about
\(MDA\) and \(MDI\) variable importances. Third, the definition of
\(MDA\) variable importance involves a permutation step that is
difficult to analyze mathematically. The above results are
mathematically amenable due to the noising-up procedure adopted allowing
for a quite elegant analysis of the variable importance. One possible
approach to dealing with this issue may be to adopt the above definition
of variable importance in different random forest settings and see if
similar results to Ishwaran (2007) may be derived. Another approach,
which is the one more or less taken by Ishwaran (2007), is to to find a
mathematically tractible definition of variable importance which is
approximately the MDA variable importance measure and proceed with
analysis from there.

\section{Theoretical Analysis of MDI Variable
Importance}\label{theoretical-analysis-of-mdi-variable-importance}

Up to now in this chapter, we have been discussing theoretical results
concerning the \(MDA\) variable importance. Louppe (2014) has explored
some of the theory for \(MDI\) variable importance of forest ensembles.

In order to discuss the following theoretical results regarding MDI
variable importance, we make some modifications to the setting we are
working in. Assume that we are working in a finite probability space
{[}this assumption is necessary primarily for technical reasons{]} and
that observations \(Y,X_1,\ldots,X_p\) in the training set \(Z\) are
categorical variables. Recall that if a forest ensemble is grown by
choosing splits and variables maximizing the decrease in nodal impurity
where \(i(s,t)\) denotes the impurity measure used, then the MDI
variable importance of the variable \(X_j\) is defined to be
\[VI(X_j)=\frac{1}{B}\sum_{b=1}^B \sum_{t\in T_b} \mathbb{I}(j_t=j)p(t)\Delta i(s,t),\]
where \(p(t)\) is the proportion of samples which reach the node \(t\)
in the tree \(T_b\). Note that MDI variable importance is the average
over the bootstrap resamples of the sum of the weighted decrease in
impurity over the nodes of the tree using \(X_j\) as the splitting
variable.

Rather than working with random forests, we will work with an infinite
ensemble of totally randomized and fully developed trees. Totally
randomized and fully developed trees are a variant of forest ensembles
given by the following. A totally randomized and fully developed tree is
a decision tree in which each node \(t\) is partioned using a variable
\(X_j\) picked uniformly at random among those not yet used at the
parent nodes of \(t\), where each node \(t\) is split into \(|X_j|\)
sub-trees, i.e., one for each possiblevalue of \(X_j\), and where the
recursive construction process halts only when all \(p\) variables have
been used along the current branch.

Using an impurity measure called Shannon entropy, Louppe (2014) is able
to show that only relevant variables (that is variables whose
information content, conditional on the other variables, is useful
towards explaining the response \(Y\)) have non-zero infinite sample
size variable importance. Furthermore, this means that irrelevant
variables (which are variables whose information content, conditional on
the other variables, is not useful towards explaining the response
\(Y\)) have a infinite sample size variable importance of zero. In fact,
in this context, a variable is irrelevant if and only if it has an
infinite sample size variable importance of zero. Generalizing from
Shannon entropy to other impurity measures, for the GINI index (used in
commonly in classification problems) and variance (used in regression
via MSE), only irrelevant variables result in no decrease in nodal
impurity. We would like to reiterate that these results concerning
irrelvant variables are valid particularly in the setting of infinite
ensembles of totally randomized and fully developed trees.

In the random forest setting, the results discussed in the previous
paragraph may not be valid. In the random forest setting, a key step in
the construction of trees is the randomization step in which we choose
at random a subset of size \(m\leq p\) of the variables to choose the
next split from. If \(m>1\), then it is possible for some variables to
never be chosen at a node \(t\), since there will always be other
variables which have a larger decrease in nodal impurity. This results
in trees in which the variables with the largest reduction in nodal
impurity are centered in the trunk of the tree with variables with
smaller reduction in nodal impurity are pushed to leaves, or otherwise
not chosen to split on. This is an undesirable property to have since
this can result in an ensemble which inadequately explores the feature
space and is thus more consistently biased towards some variables over
others, even if other variables can provide only slightly lower
reductions in nodal impurity. As a consequence, if \(m>1\), then it is
not the case that a variable is irrelevant if and only if it has a
variable importance of zero. One implication is that in data sets where
there are correlated variables, random forest variable importance
measures may result in variable importances which do not reflect the
actually importance of variables in explaining the response \(Y\). In
particular, it could be the case that if \(X_j\) and \(X_{j'}\) are two
correlated variables, then even if both \(X_j\) and \(X_{j'}\) are
important in explaining the response \(Y\), it may be the case that
\(X_j\) is consistently chosen as the splitting variable over \(X_{j'}\)
if the two variables are both possible splitting variables.

\section{Dealing with Bias in Random Forest Variable
Importance}\label{dealing-with-bias-in-random-forest-variable-importance}

In the previous two sections of this chapter we discussed theoretical
results pertaining to MDA and MDI variable importance measures. For MDA
variable importance, a consistent ensemble with orthogonal trees have a
nice asymptotic form in the framework developed in Ishwaran (2007). On
the otherhand, with MDI variable importance, if the trees are grown in
the random forest setting, then there are issues of the variable
importance not capturing the actual importance of the variables in the
training data. This could particularly arise as an issue in data sets
with correlated variables where equally important variables receive
different variable importances due to bias introduced by tree
construction process. In using the term `bias,' we would like to
emphasise that this is not bias in the sense of biased estimators, but
bias in the sense of which variables are chosen to be split upon in the
tree growing process. Bias in the variable importance measures of random
forests is concerning as we would like to be able to use random forests
for inferential statistics. A couple of methods have been proposed to
construct more reliable variable importance measures, which we discuss
in this section.

\subsection{Bias in Random Forest Variable Importance
Measures}\label{bias-in-random-forest-variable-importance-measures}

Bias in random forest variable importance measures was first
substantively analyzed in Strobl et al. (2007). Strobl et al. (2007) ran
a simulation study comparing MDI variable importance using the GINI
index, MDA variable importance, and a conditional variable importance
measure in a classification setting. The conditional variable importance
measure used by Strobl et al. (2007) is a variable importance method
similar to MDA variable importance, but utilizing an ensemble of
conditional inference trees. We will define conditional inference trees
and the conditional variable importance measures below. For now, note
that Strobl et al. (2007) found that MDI variable importance using the
GINI index and MDA variable importance tends to be biased from the
expected variable importance value within their simulation. They also
found that subsampling without replacement tends to reduce the bias in
variable importance in comparison to using bootstrap resampling with
replacement. In their analysis, Strobl et al. (2007) claim that the GINI
index tends to favor variables with more potential cutpoints. Hence the
MDI variable importance measure using the GINI index is biased towards
variables with more potential cutpoints. Strobl et al. (2007) also found
that in using the MDA variable importance measure, since variables that
are split upon closer to the root node affect the predictive accuracy of
the ensemble more than variables split upon in leaves, the MDA variable
importance measure will be affected by the variable selection bias of
individual trees. Furthermore, Strobl et al. (2008) find that correlated
variables tend to be overselected in the tree growing process. In order
to utilize random forest variable importance measures as inferential
tools, several methods have been proposed to deal with the issue of
variable selection bias in random forests.

\subsection{Conditional Variable
Importance}\label{conditional-variable-importance}

The MDA variable importance measure can be viewed in the context of
permutation tests and hypothesis testing. In particular, we can consider
that the MDA variable importance measure operates on the null hypothesis
that permuting the values of the variable \(X_j\) has no affect on the
predictive accuracy of the random forest predictor. This corresponds to
the null hypothesis that the variable \(X_j\) is independent of the
response \(Y\) and the variables \(X_1,\ldots,X_p\). If the predictive
accuracy of the random forest suffers from permuting \(X_j\), that is if
the \(VI(X_j)\) is nonzero, then if also \(X_j\perp X_{-j}\), it would
follow that \(Y\) and \(X_j\) are not independent. However, it is
important to note that the null hypothesis under which the MDA variable
importance measure operates is: \[H_0:X_j \perp Y,X_{-j}.\] If on the
otherhand, \(X_j\) and \(X_{-j}\) are not independent, then \(VI(X_j)\)
will be biased by the correlation between \(X_j\) and \(X_{-j}\). It
could be that \(Y\) and \(X_j\) are not independent or that \(Y\) and
\(X_j\) are independent, but in either case, if the predictor variables
are correlated, we see that \(VI(X_j)\) will not accurately capture the
importance of \(X_j\). Strobl et al. (2008) propose a permutation scheme
corresponding to the null hypothesis \[H_0:(X_j\perp Y)|X_{-j},\] that
is that \(X_j\) is independent of \(Y\) conditional on the \(X_{-j}\).
Such a permutation scheme would take into account the correlation
structure of the predictor variables and would allow for a less biased
variable importance measure so long as the individual trees within the
ensembles do not exhibit large variable selection bias.

Strobl et al. (2008) propose defining a grid within which values of
\(X_j\) are permuted according to partitions of the feature space given
by each tree. While Strobl et al. (2008) propose using unbiased
conditional inference trees from Hothorn, Hornik, \& Zeileis (2006) to
determine the partition, it is possible to use partitions given by CART
trees. The algorithm is presented as follows.
\begin{algorithm}
    \caption{Conditional Variable Importance} \label{conditional variable importance}
      \begin{algorithmic}[1]
        \State Grow a forest ensemble $\{T_b\}_{b=1}^B$
          \For{ $j=1,\ldots,p$ }
            \For{ $b=1,\ldots,B$ }
            \State Compute $RSS(T_b,Z_b)$.
            \For{ all variables $X_i$ to be conditioned on }
            \State Extract cutpoints that split $X_i$ in the tree $T_b$ and create a grid $\mathcal{G}$ by bisecting the sample space in each cutpoint.
            \EndFor
            \State Within the grid $\mathcal{G}$, permute the values of $X_j$ and denote the permuted resample by $Z_b^j$.
            \State Compute $RSS(T_b,Z_b^j)$.
            \State Compute the conditional variable importance of $X_j$ in the tree $T_b$ to be $VI_b(X_j)=\frac{1}{|Z_b|}\left(RSS(T_b,Z_b)-RSS(T_b,Z_b^j)\right)$.
            \EndFor
            \State Compute the conditional variable importance of $X_j$ in the forest ensemble to be $VI(X_j)=\frac{1}{B}\sum_{b=1}^B VI_b(X_j)$.
          \EndFor
      \end{algorithmic}
  \end{algorithm}
To determine the variables \(X_i\) to be conditioned on Strobl et al.
(2008) suggest including only variables whose empirical correlation with
the variable \(X_j\) suggests some reasonable threshold.

\subsection{Infforest Variable
Importance}\label{infforest-variable-importance}

While simulation results of Strobl et al. (2008) suggest that the
conditional variable importance is more effective at capturing the
importance of each variable \(X_j\) compared to the marginal approach
followed in MDA, they note that conditional variable importance cannot
completely eliminate preference for correlated variables in random
forests. Owens (2017) instead suggests a partition then permute scheme
which produces a sampling distribution for the variable importance of
each variable \(X_j\). Once a sampling distribution for \(VI(X_j)\) has
been obtained, hypothesis testing and other sorts of statistical
inference can proceed. As the Infforest variable importance is quite
complicated, some explanation is needed.

After growing a forest ensemble \(\{T_b\}_{b=1}^B\), infforest variable
importance values are computed as follows. First, for each
\(b=1,\ldots,B\) and each variable \(X_j\), grow a tree
\(X_j\sim X_1,\ldots,X_{j-1},X_{j+1},\ldots,X_p\) using all \(p-1\)
predictors using the in-bag sample used to grow \(T_b\). Denote this
tree by \(T_j^{b*}\). The rows of \(\overline{Z}_b\) are permuted within
the partitions of the feature space determined by \(T_j^{b*}\). The
difference in predictive accuracy of \(T_b\) before and after the OOB
sample \(\overline{Z}_b\) has been permuted is the infforest variable
importance for the variable \(X_j\) in the tree \(T_b\). Owens (2017)
suggests then using the sampling distribution of \(VI(X_j)\) obtained
from the infforest partition-permute scheme to test the null hypothesis
that the variable importance of \(X_j\) is zero. On the other hand, if
we let \(VI_b(X_j)\) denote the infforest variable importance of the
variable \(X_j\) in the tree \(T_b\), then we could define the infforest
variable importance of \(X_j\) in the forest ensemble
\(\{T_b\}_{b=1}^B\) to be \[VI(X_j)=\frac{1}{B}\sum_{b=1}^B VI_b(X_j).\]
In particular, this corresponds to the mean of the sampling distribution
of the infforest variable importances of \(X_j\) over the trees \(T_b\).

Originally, Owens (2017) developed for random forests using CART trees
as base learners. However, the infforest variable importance could
certainly be used with random forest variants such as conditional
inference forests used in Strobl et al. (2008). Another possible
modification of the infforest variable importance algorithm is to only
grow an auxilary tree predicting
\(X_j\sim X_{\alpha_1},\ldots,X_{\alpha_k}\), where
\(X_{\alpha_1},\ldots,X_{\alpha_k}\) are variables whose correlation
with \(X_j\) exceeds some reasonable threshold. This would combine the
suggestion in Strobl et al. (2008) of considering the correlation
structure of the predictor variables in setting up the grid for the
conditional variable importance with the infforest partition-permute
scheme.

One issue with infforest variable importance is the steep computational
cost of the algorithm. The algorithm grows an auxilary tree
\(T_{j}^{b*}\) for each variable \(X_j\) and bootstrap resample \(b\)
and then proceeds to permute the OOB sample with respect to each
\(T_j^{b*}\). Even implementing the infforest variable importance
algorithm on small bootstrap resamples of around several hundred is
computationally intensive.
\begin{algorithm}
    \caption{Infforest Variable Importance} \label{infforest variable importance}
      \begin{algorithmic}[1]
        \State Grow a forest ensemble $\{T_b\}_{b=1}^B$
          \For{ $j=1,\ldots,p$ }
            \For{ $b=1,\ldots,B$ }
              \State Compute $RSS(T_b,\overline{Z}_b)$.
              \State Grow a tree $T_j^{b*}$ predicting $X_j\sim X_{-j}$ using in-bag sample.
              \State Permute rows of the OOB sample $\overline{Z}_b$ with respect to the partitions of the feature space from $T_j^{b*}$ to obtain the permuted OOB sample $\overline{Z}_b^*$. 
              \State Compute $RSS(T_b,\overline{Z}_b^*)$.
              \State Compute the infforest variable importance of $X_j$ in $T_b$ to be $VI_b(X_j)=\frac{1}{\overline{Z}_b}\left(RSS(T_b,\overline{Z}_b)-RSS(T_b,\overline{Z}_b^*)\right)$.  
            \EndFor
            \State Test the null hypothesis that the variable importance of $X_j$ is zero using the distribution of values $VI_b(X_j)$. 
          \EndFor
      \end{algorithmic}
  \end{algorithm}
\chapter{Added Variable Plot Variable
Impotance}\label{added-variable-plot-variable-impotance}

\section{Introduction}\label{introduction-2}

In the previous chapter, we discussed theoretical properties and
observed behavior of random forest variable importance measures. In
particular, we discussed issues of bias present in the MDA variable
importance measure. Strobl et al. (2008) and Owens (2017) proposed the
conditional variable importance measure and the Infforest variable
importance measures, respectively, as methods of accounting for bias in
variable selection among correlated predictors when measuring variable
importance in a forest ensemble. Conditional variable importance and
Infforest variable importance aim at measuring the conditional
importance of a particular variable by conditionally permuting OOB data
according to some criteria measured from the data, whether that be
empirical correlation in the case of conditional variable importance, or
the partitions induced by a particular tree inthe case of Infforest
variable importance. In any case, we are primarily interested in the
conditional importance of a predictor given the information provided by
other predictors. While we would like to have a exact method of
computing the conditional importance of a predictor in a random forest
ensemble, in practice it is unclear how to construct an exact method
given the complexity of the random forest ensemble. We can, however,
estimate the effect of adding a predictor to the set of predictors that
the forest ensemble can split on via a type of diagnostic plot called
added variable plots. Added variable plots are a diagnostic plot arising
from the linear regression which estimate the effect of adding a
predictor variable to the model. In this chapter we propose a method of
measuring the importance of each predictor in a random forest ensemble
via a quantity we call the Added Variable Importance (AVI) that depends
on the added variable plot of each predictor. The AVI attempts to
estimate the effect of adding a predictor as a possible candidate in the
splitting step of the random forest ensemble. To understand AVI, we
regress from the random forest setting to the linear regression setting
and explain how added variable plots work in the linear regression
setting.

\section{Added Variable Plots in Linear
Regression}\label{added-variable-plots-in-linear-regression}

Suppose that we have data \(Z=\{(\mathbf{x}_i,Y_i)|i=1,\ldots,n\}\),
where \(\mathbf{x}_i\) is a covariate in the feature space, and that we
fit a multiple linear regression model of the form
\[\mathbf{Y}_1=\mathbf{X}\beta_1+\mathbf{W}\alpha+\mathbf{\varepsilon}_1\]
where \(\mathbf{X}\) is a \(n\times p\) matrix consisting of \(p-1\)
predictors, \(\beta=(\beta_0,\beta_1,\ldots,\beta_p)^T\),
\(\mathbf{W}=(w_1,\ldots,w_n\) is a predictor, \(\alpha\) is a scalar,
and \(\text{Var}(\mathbf{\varepsilon})=\sigma^2 I\).

Say we are interested in estimating the effect of the predictor
\(\mathbf{W}\) in the regression model \(\hat{\mathbf{Y}}_1\). We could
first fit the models
\(\hat{\mathbf{Y}}_2=\mathbf{X}\beta_2+\mathbf{\varepsilon}_2\) and
\(\hat{\mathbf{W}}=\mathbf{X}\delta+\mathbf{\varepsilon}\), where
\(\beta_2\) and \(\delta\) are coefficient vectors like \(\beta_1\) and
then plot the residuals \(\mathbf{W}-\hat{\mathbf{W}}\) against the
residuals \(\mathbf{Y}-\hat{\mathbf{Y}}_2\). The plot we obtain by
plotting \(\mathbf{W}-\hat{\mathbf{W}}\) against
\(\mathbf{Y}-\hat{\mathbf{Y}}_2\) is called the added variable plot of
\(\mathbf{W}\) and measures the effect of \(\mathbf{W}\) on
\(\mathbf{Y}\) once we have adjusted for the effect of \(\mathbf{X}\) on
\(\mathbf{W}\) and \(\mathbf{Y}\), respectively. In particular, the
residuals that we plot for the added variable plot is the portion of
\(\mathbf{W}\) unexplained by \(\mathbf{X}\) on the \(x\)-axis and the
portion of \(\mathbf{Y}\) unexplained by
\(\hat{\mathbf{Y}}_2=\mathbf{X}\beta_2+\mathbf{\varepsilon}\) on the
\(y\)-axis.

In the linear regression setting the added variable plot has some nice
properties. In particular, let \(\alpha_{AVP}\) denote the estimate of
the slope from regressing \(\mathbf{Y}-\hat{\mathbf{Y}}_2\) on
\(\mathbf{W}-\hat{\mathbf{W}}\). Then with some linear algebra it can be
shown that \(\alpha_{AVP}\) is equal to the least-squares estimate of
\(\alpha\) from
\(\hat{\mathbf{Y}}_1=\mathbf{X}\beta_1+\mathbf{W}\alpha+\mathbf{\varepsilon}_1\).
For full details see Sheather (2009) and Cook \& Weisberg (1982). Hence
if we want to estimate the linear effect of the variable \(\mathbf{W}\)
on the linear regression \(\hat{\mathbf{Y}}_1\), we can construct and
visually inspect the trend of the added variable plot of \(\mathbf{W}\).

{[}Provide example of added variable plot{]}

\section{Added Variable Plots in Random
Forests}\label{added-variable-plots-in-random-forests}

While added variable plots in the linear regression setting allows us to
conditionally estimate the effect of the predictor on the response, the
picture for more complex regression functions such as random forests is
more complicated. In particular, if the relationship between the
response and predictors is non-linear, or we are using a statistical
learning method such as a random forest, then we do not have the nice
linear algebra that undergirds the interpretation of added variable
plots of linear regression models. However, we can still construct a
variant of added variable plots for random forests.

Suppose we are interested in the added variable effect of variable
\(X_k\). Let \(X_{-k}\) denote the set of predictors excluding \(X_k\).
As Rendahl (2008) notes, in general, for black box statistical learning
methods such as random forests the only appropriate added variable plot
we can form is to plot \(Y-\mathbb{E}(Y|X_{-k}\) against
\(\mathbb{E}(Y|X)-\mathbb{E}(Y|X_{-k})\). That is, we can plot residuals
of the model without the predictor \(X_k\) against the difference in
predictions between the full model and the model with \(X_k\) removed.
In the context of random forests, to obtain the added variable plot of
the predictor \(X_k\), we would plot
\[(\hat{\theta}_{RF}(Y|X)-\hat{\theta}_{RF}(Y|X_{-k}), Y-\hat{\theta}_{RF}(Y|X_{-k})).\]

{[}Add example here{]}

Depending on the relationship between the predictors and the response,
there are several outputs we might expect from the added variable plots
of the predictors in the random forest. If the predictor \(X_k\) is
simply noise, i.e., if a predictor is uninformative with respect to the
response, then we expect that \(\hat{\theta}_{RF}(Y|X)\) and
\(\hat{\theta}_{RF}(Y|X_{-k})\) to form similar predictions of \(Y\)
given \(X=x\) and \(X_{-k}=x_{-k}\), respectively. In this scenario
then, suppose for a moment that \(\hat{\theta}_{RF}(Y|X)\) and
\(\hat{\theta}_{RF}(Y|X_{-k})\) are consistent estimators of
\(\mathbb{E}(Y|X)\) and \(\mathbb{E}(Y|X_{-k})\), respectively. Then by
the Law of Total Expectation, we would expect that asymptotically
\(\mathbb{E}(\hat{\theta}_{RF}(Y|X))=Y\) and
\(\mathbb{E}(\hat{\theta}_{RF}(Y|X_{-k}))=Y\). Hence in the limit, by
linearity of expectation,
\[\mathbb{E}(\hat{\theta}_{RF}(Y|X)-\hat{\theta}_{RF}(Y|X_{-k}))=\mathbb{E}(\hat{\theta}_{RF}(Y|X))-\mathbb{E}(\hat{\theta}_{RF}(Y|X_{-k}))=Y-Y=0.\]
Similarly, \[\mathbb{E}(Y-\hat{\theta}_{RF}(Y|X_{-k}))=Y-Y=0.\] Hence if
a predictor \(X_k\) is uninformative, provided we have grown adequetely
accurate forest ensembles to predict \(Y\) given \(X\) and \(Y\) given
\(X_{-k}\), respectively, then we expect that the the added variable
plot of \(X_k\) to be radially centered about the origin or to otherwise
be arranged in a line with a slope and \(y\)-intercept of zero.

On the other hand, suppose that the predictor \(X_k\) is informative
with respect to the response. Then we would expect that when properly
tuned the random forest ensemble \(\hat{\theta}_{RF}(Y|X)\) would
provide reasonably accurate predictions of \(Y\). We would also expect
that the forest ensemble \(\hat{\theta}_{RF}(Y|X_{-k})\) would suffer in
predictive performance due to the lack of information from \(X_k\). Then
in the limit, we would expect that
\[(\hat{\theta}_{RF}(Y|X)-\hat{\theta}_{RF}(Y|X_{-k})\neq 0 \text{ and } Y-\hat{\theta}_{RF}(Y|X_{-k}))\neq 0.\]
That is, we would expect the trend of the added variable plot of \(X_k\)
to be non-zero. In other words, in the case of informative predictors,
we would expect to be correlation between
\(\hat{\theta}_{RF}(Y|X)-\hat{\theta}_{RF}(Y|X_{-k})\) and
\(Y-\hat{\theta}_{RF}(Y|X_{-k})\). The departure from a trend of
approximately zero would depend on the relative informativeness of
\(X_k\). If \(X_k\) is strongly informative then we would expect the
difference between \(Y\) and \(\hat{\theta}_{RF}(Y|X_{-k})\) and the
difference between \(\hat{\theta}_{RF}(Y|X)\) and
\(\hat{\theta}_{RF}(Y|X_{-k})\) to often be large. If \(X_k\) is weakly
informative we would expect the two differnces to often be of smaller
magnitude than when \(X_k\) is strongly informative. Hence visually the
trend of the added variable plot of predictors used to grow a random
forest ensemble offer a method of gauging the informativeness of
different predictors. We note that in the above argument that we made an
appeal to the random forest estimator \(\hat{\theta}_{RF}(Y|X)\) being a
consistent estimator of \(\mathbb{E}(Y|X)\). As discussed earlier, the
random forest algorithm as introduced and implemented by Breiman and
collaboraters has not yet been shown to be consistent. However,
empirically the random forest algorithm often offers good predictions of
the response, so we expect that the added variable plot as applied to
the original random forest algorithm to have the properties discussed
above. We also note that there are random forest variants such as Causal
Forest as introduced by Wager \& Athey (2017) which have been shown to
be consistent estimators of \(\mathbb{E}(Y|X)\). Hence in such settings
we expect that our discussion of added variable plots for forest
ensembles to be fully valid.

One setting in which added variable plots for random forests are useful
is in dealing with correlated predictors. As Strobl et al. (2008) notes,
the random forest algorithm can have difficulty determining the relative
importance of correlated predictors due to masking effects. Suppose
\(X_j\) and \(X_k\) are correlated predictors with \(X_j\) only weakly
informative to the response. Then if both \(X_j\) and \(X_k\) are
candidate splitting variables, \(X_k\) could be chosen over \(X_j\) as
the splitting variable since \(X_k\) may seem to be informative due to
the correlation between \(X_k\) and \(X_j\). The better split would have
been found over \(X_j\), but the greedy nature of the CART algorithm
means the algorithm does not look forward at possible splits further
down the tree if \(X_j\) is chosen over \(X_k\). Note that the tree
grown using \(X_k\) at that particular node would likely have lower
predictive performance than the tree grown using \(X_j\) at that node.
Hence WLOG suppose the subset
\(\{X_1,\ldots,X_m\}\subseteq \{X_1,\ldots,X_p\}\) of our predictors are
correlated where \(m\leq p\). If \(X_j\in \{X_1,\ldots,X_m\}\) is not
informative to the response, then we would expect
\(\hat{\theta}_{RF}(Y|X)\) and \(\hat{\theta}_{RF}(Y|X_{-j})\) to
provide similar predictions, while if \(X_k\in \{X_1,\ldots,X_m\}\) is
informative to the response, we would expect that
\(\hat{\theta}_{RF}(Y|X_{-k})\) will suffer a decrease in predictive
performance in comparison to \(\hat{\theta}_{RF}(Y|X)\). Then we would
expect the added variable plot for \(X_j\) to be a mass of points
centered about the origin without a trend while the added variable plot
for \(X_k\) should have some sort of trend whose shape depends on the
informativeness of \(X_k\) and the predictive performance of
\(\hat{\theta}_{RF}(Y|X)\) and \(\hat{\theta}_{RF}(Y|X_{-k})\). This is
of course despite the fact that, depending on the correlation structure
and relative informativeness of \(X_k\), that \(X_j\) may be chosen over
\(X_k\) when \(X_j\) and \(X_k\) are both candidate splitting variables
in the tree growing process. The full random forest model
\(\hat{\theta}_{RF}(Y|X)\) may be biased in how the splitting variables
are chosen due to the correlation structure of \(\{X_1,\ldots,X_m\}\) as
discussed in Strobl et al. (2008). However, absent the choice of \(X_k\)
in the random forest model \(\hat{\theta}_{RF}(Y|X_{-k})\), the less
informative split on \(X_j\) has an increased probability of being
chosen when growing the forest ensemble \(\hat{\theta}_{RF}(Y|X_{-k})\)
than in the full forest ensemble \(\hat{\theta}_{RF}(Y|X)\). Hence the
random forest model \(\hat{\theta}_{RF}(Y|X_{-k})\) would have a
decrease in predictive in performance due to the loss of information in
the predictor \(X_k\), but also due to the increased probability of
irrelevant predictors being chosen as the splitting variables due to
exclusion of \(X_k\).

\section{Added Variable Plot
Importance}\label{added-variable-plot-importance}

In the previous section, we discussed the use of added variable plots
for random forests including in settings where a subset of the
predictors are correlated. Variable importance measures such as MDA
variable importance and MDI variable importance have difficulties in
dealing with correlated predictors. As discussed in the previous
chapter, Strobl et al. (2008) proposed the conditional variable
importance measure while Owens (2017) proposed the Infforest variable
importance measure as methods that try to account for the correlation
structure of the predictors when measuring the importance of predictors
in the forest ensemble. In this section we propose the Added Variable
Plot Importance (AVPI) measure as an alternative variable importance
measure to conditional variable importance and Infforest variable
importance measures in accounting for correlated predictors when
measuring variable importance.

To motivate the construction of AVPI, we return briefly to added
variable plots in the linear regression setting. As mentioned earlier,
when we construct an added variable plot for a predictor \(\mathbf{W}\)
in the linear regression model, the linear trend of the added variable
plot for \(\mathbf{W}\) corresponds to the least squares estimate of the
linear effect \(\alpha\) of \(\mathbf{W}\) in the full model
\(\mathbf{Y}_1=\mathbf{X}\beta_1+\mathbf{W}\alpha+\mathbf{\varepsilon}_1\).
Hence while we could simply visually inspect the trend of the added
variable plot for \(\mathbf{W}\), we could also run least squares
regression on the added variable plot for \(\mathbf{W}\) to obtain an
estimate for \(\alpha\) in the regression model \(\mathbf{Y}_1\). The
size and sign of the \(\hat{\alpha}\) from running a regression model on
the added variable plot of \(\mathbf{W}\) then indicates the relative
importance and effect of \(\mathbf{W}\) on the response \(\mathbf{Y}\).

In the random forest setting, we would like to have a similar procedure
to determine the importance of the variable \(X_k\) with respect to the
response \(Y\) once we have accounted for the effect of the other
predictors, but as noted before, we are restricted to plotting the added
variable plot of \(X_k\) using the random forest algorithm as
\[(\hat{\theta}_{RF}(Y|X)-\hat{\theta}_{RF}(Y|X_{-k}), Y-\hat{\theta}_{RF}(Y|X_{-k})).\]
As discussed in the previous section, depending on the relative
informativeness of the predictor \(X_k\) there may or may not be a trend
in the added variable plot for \(X_k\) which we could then attempt to
model. Let \(W_k=Y-\hat{\theta}_{RF}(Y|X_{-k})\) and
\(U_k=\hat{\theta}_{RF}(Y|X)-\hat{\theta}_{RF}(Y|X_{-k})\). We propose
to train a bagged forest ensemble \(\hat{\theta}_{BF}(W_k|U_k)\) and
computing the MDA variable importance of \(U_k\) in the bagged forest
ensemble. We call the MDA variable importance of \(U_k\) in the bagged
forest ensemble \(\hat{\theta}_{BF}(W_k|U_k)\) the added variable plot
importance (AVPI) of \(X_k\) and denote this quantity by
\(VI_{AVP}(X_k)\). Our choice of using a bagged forest to predict
\(W_k\) given \(U_k\) is motivated by the fact that unlike in the linear
regression setting, there is not a clear parametric model with which to
predict \(W_k\) given \(U_k\). The bagged forest makes few assumptions
on the form of the true regression function \(W_k=f(U_k)+\varepsilon\)
while also providing a metric in the form of MDA variable importance to
assess the importance of \(U_k\) in predicting \(W_k\) once we have
grown the ensemble \(\hat{\theta}_{BF}(W_k|U_k)\). Furthermore, the AVPI
of \(X_k\) should reflect the relative importance of \(X_k\) with
respect to the response \(Y\) given that the trend of the added variable
plot, i.e.~the degree to which \(Y-\hat{\theta}_{RF}(Y|X_{-k})\) and
\(\hat{\theta}_{RF}(Y|X)-\hat{\theta}_{RF}(Y|X_{-k})\) are correlated,
visually indicates the informativeness of \(X_k\).

The purpose of the AVPI of \(X_k\) is to provide a quantitative measure
of the importance of the predictor \(X_k\) with respect to \(Y\) once we
have taken into account the loss in predictive performance when \(X_k\)
is removed as a possible splitting variable. In particular, given that
the added variable plots for random forests should reflect the
informativeness of correlated predictors, the added variable plot
importance for correlated predictors should be able to more accurately
reflect the importance of correlated predictors with respect to the
response than the MDA variable importance ran on the predictors in the
full model \(\hat{\theta}_{RF}(Y|X)\).
\begin{algorithm}
    \caption{Added Variable Plot Importance (AVPI)} \label{added variable importance}
      \begin{algorithmic}[1]
          \State Grow the random forest ensemble $\hat{\theta}_{RF}(Y|X)$ predicting $Y$ using the full set of predictors. 
          \For{ $k=1,\ldots,p$ }
            \State Grow the random forest ensemble $\hat{\theta}_{RF}(Y|X_{-k})$ predicting $Y$ using the full set of predictors minus the predictor $X_k$.
            \State Compute $U_k=\hat{\theta}_{RF}(Y|X)-\hat{\theta}_{RF}(Y|X_{-k})$ and $W_k=Y-\hat{\theta}_{RF}(Y|X_{-k})).$
            \State Grow the bagged forest ensemble $\hat{\theta}_{BF}(W_k|U_k)$ predicting $W_k$ using $U_k$. 
            \State Compute the added variable plot importance of $X_k$ to be the MDA variable importance of $U_k$: $VI_{AVP}(X_k)=VI_{MDA}(U_K)$.
          \EndFor
      \end{algorithmic}
  \end{algorithm}
\section{Extensions of Added Variable Plot
Importance}\label{extensions-of-added-variable-plot-importance}

Once we have obtained added variable plot importance values there are
couple directions we can extend our framework. Once again suppose we
have grown our full random forest ensemble \(\hat{\theta}_{RF}(Y|X)\)
and also the random forest ensemble \(\hat{\theta}_{RF}(Y|X_{-k})\). Let
\(U_k=\hat{\theta}_{RF}(Y|X)-\hat{\theta}_{RF}(Y|X_{-k})\) and
\(W_k=Y-\hat{\theta}_{RF}(Y|X_{-k})).\)

Once we have computed the AVPI of \(X_k\), \(VI_{AVP}(X_k)\), we can
take a simulation approach to generating a sampling distribution for
\(VI_{AVP}(X_k)\). In particular, to generate a sampling distribution
for the AVPI of \(X_k\), permute \(U_k\) to obtain \(U_k^*\). Then grow
the bagged forest ensemble \(\hat{\theta}_{BF}(W_k|U_k^*)\) predicting
\(W_k\) using \(U_k^*\), and compute the MDA variable importance of
\(U_k^*\) to obtain \(VI_{AVP}^*(X_k)\) as the permuted AVPI for
\(X_k\). After having permuted \(U_k\) and computed \(VI_{AVP}^*(X_k)\)
for enough iterations to generate a sampling distribution for
\(VI_{AVP}(X_k)\). We can then compute a two-sided p-value using the
original \(VI_{AVP}(X_k)\) as the observed test statistic. If we compute
p-values for each predictor, then we could proceed to a hypothesis
testing framework if we believe that the simulation to generate a
sampling distribution for the AVPI of each predictor was successful. Of
course depending on the number of predictors in the dataset and the
relationship between the predictors, it may be necessary to control for
multiple comparisons. In our opinion, as simulations in the next chapter
will indicate, hypothesis testing using the AVPI is likely more
sensitive to type-I errors than type-II errors, so an adjustment such as
the Bonferroni correction may be appropriate.

We also note that while generating adequete sampling distributions of
\(VI_{AVP}(X_k)\) for each predictor \(X_k\) is computationally
expensive, each step of the process from growing the full ensemble
\(\hat{\theta}_{RF}(Y|X)\) to growing each
\(\hat{\theta}_{RF}(Y|X_{-k})\) to permuting \(U_k\) and growing
\(\hat{\theta}_{BF}(W_k|U_k^*)\) to compute \(VI_{AVP}^*(X_k)\) can be
coded to run in parallel. So performance gains in generating good
sampling distributions of \(VI_{AVP}(X_k)\) for each predictor \(X_k\)
are easily attainable.

\chapter*{Conclusion}\label{conclusion}
\addcontentsline{toc}{chapter}{Conclusion}

If we don't want Conclusion to have a chapter number next to it, we can
add the \texttt{\{-\}} attribute.

\textbf{More info}

And here's some other random info: the first paragraph after a chapter
title or section head \emph{shouldn't be} indented, because indents are
to tell the reader that you're starting a new paragraph. Since that's
obvious after a chapter or section title, proper typesetting doesn't add
an indent there.

\appendix

\chapter{The First Appendix}\label{the-first-appendix}

This first appendix includes all of the R chunks of code that were
hidden throughout the document (using the \texttt{include\ =\ FALSE}
chunk tag) to help with readibility and/or setup.

\textbf{In the main Rmd file}
\begin{Shaded}
\begin{Highlighting}[]
\CommentTok{# This chunk ensures that the thesisdown package is}
\CommentTok{# installed and loaded. This thesisdown package includes}
\CommentTok{# the template files for the thesis.}
\ControlFlowTok{if}\NormalTok{(}\OperatorTok{!}\KeywordTok{require}\NormalTok{(devtools))}
  \KeywordTok{install.packages}\NormalTok{(}\StringTok{"devtools"}\NormalTok{, }\DataTypeTok{repos =} \StringTok{"http://cran.rstudio.com"}\NormalTok{)}
\ControlFlowTok{if}\NormalTok{(}\OperatorTok{!}\KeywordTok{require}\NormalTok{(thesisdown))}
\NormalTok{  devtools}\OperatorTok{::}\KeywordTok{install_github}\NormalTok{(}\StringTok{"ismayc/thesisdown"}\NormalTok{)}
\KeywordTok{library}\NormalTok{(thesisdown)}
\end{Highlighting}
\end{Shaded}
\textbf{In Chapter \ref{ref-labels}:}

\chapter{The Second Appendix, for
Fun}\label{the-second-appendix-for-fun}

\backmatter

\chapter*{References}\label{references}
\addcontentsline{toc}{chapter}{References}

\markboth{References}{References}

\noindent

\setlength{\parindent}{-0.20in} \setlength{\leftskip}{0.20in}
\setlength{\parskip}{8pt}

\hypertarget{refs}{}
\hypertarget{ref-biau2015a}{}
Biau, G., \& Scornet, E. (2016). A random forest guided tour. \emph{E.
TEST}, \emph{25}(2), 197--227.

\hypertarget{ref-cook1982}{}
Cook, D. R., \& Weisberg, S. (1982). \emph{Residuals and influence in
regression}. London: Chapman; Hall.

\hypertarget{ref-efron2014}{}
Efron, B. (2014). Estimation and accuracy after model selection.
\emph{Journal of the American Statistical Association}, \emph{109}(507),
991--1007. \url{http://doi.org/doi.org/10.1080/01621459.2013.823775}

\hypertarget{ref-hothorn2006}{}
Hothorn, T., Hornik, K., \& Zeileis, A. (2006). Unbiased recursive
partitioning: A conditional inference framework. \emph{Journal of
Computational and Graphical Statistics}, \emph{15}(3), 651--674.
\url{http://doi.org/10.1198/106186006X133933}

\hypertarget{ref-ishwaran2007}{}
Ishwaran, H. (2007). Variable importance in binary regression trees and
forests. \emph{Electron J Stat}, 519--537.

\hypertarget{ref-ishwaran2015}{}
Ishwaran, H. (2015). The effect of splitting on random forests.
\emph{Mach. Learn.}, \emph{99}(1), 75--118.
\url{http://doi.org/10.1007/s10994-014-5451-2}

\hypertarget{ref-louppe2014}{}
Louppe, G. (2014). \emph{Understanding random forests from theory to
practice} (PhD dissertation). University of Liège.

\hypertarget{ref-mentch2016b}{}
Mentch, L., \& Hooker, G. (2016a). Formal hypothesis tests for additive
structure in random forests. Retrieved from
\url{http://arxiv.org/abs/1406.1845}

\hypertarget{ref-mentch2016a}{}
Mentch, L., \& Hooker, G. (2016b). Quantifying uncertainty in random
forests via confidence intervals and hypothesis tests. \emph{Journal of
Machine Learning Research}, \emph{17}(26), 1--41. Retrieved from
\url{http://jmlr.org/papers/v17/14-168.html}

\hypertarget{ref-owens2017}{}
Owens, A. (2017). \emph{INFFOREST variable importance for random
forests} (Undergraduate Thesis). Reed College.

\hypertarget{ref-rendahl2008}{}
Rendahl, A. K. (2008). \emph{Graphical methods of determining predictor
importance and effect} (PhD dissertation). University of Minnesota.

\hypertarget{ref-sheather2009}{}
Sheather, S. (2009). \emph{A modern approach to regression with r}. New
York: Springer-Verlag.

\hypertarget{ref-strobl2008}{}
Strobl, C., Boulesteix, A.-L., Kneib, T., Augustin, T., \& Zeileis, A.
(2008). Conditional variable importance for random forests. \emph{BMC
Bioinformatics}, \emph{9}(1), 307.
\url{http://doi.org/10.1186/1471-2105-9-307}

\hypertarget{ref-strobl2007}{}
Strobl, C., Boulesteix, A.-L., Zeileis, A., \& Hothorn, T. (2007). Bias
in random forest variable importance measures: Illustrations, sources
and a solution. \emph{BMC Bioinformatics}, \emph{8}(1), 25.
\url{http://doi.org/10.1186/1471-2105-8-25}

\hypertarget{ref-wager2017}{}
Wager, S., \& Athey, S. (2017). Estimation and inference of
heterogeneous treatment effects using random forests. \emph{Journal of
the American Statistical Association}, \emph{0}(ja), 0--0.
\url{http://doi.org/10.1080/01621459.2017.1319839}


% Index?

\end{document}

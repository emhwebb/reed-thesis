% This is the Reed College LaTeX thesis template. Most of the work
% for the document class was done by Sam Noble (SN), as well as this
% template. Later comments etc. by Ben Salzberg (BTS). Additional
% restructuring and APA support by Jess Youngberg (JY).
% Your comments and suggestions are more than welcome; please email
% them to cus@reed.edu
%
% See http://web.reed.edu/cis/help/latex.html for help. There are a
% great bunch of help pages there, with notes on
% getting started, bibtex, etc. Go there and read it if you're not
% already familiar with LaTeX.
%
% Any line that starts with a percent symbol is a comment.
% They won't show up in the document, and are useful for notes
% to yourself and explaining commands.
% Commenting also removes a line from the document;
% very handy for troubleshooting problems. -BTS

% As far as I know, this follows the requirements laid out in
% the 2002-2003 Senior Handbook. Ask a librarian to check the
% document before binding. -SN

%%
%% Preamble
%%
% \documentclass{<something>} must begin each LaTeX document
\documentclass[12pt,twoside]{reedthesis}
% Packages are extensions to the basic LaTeX functions. Whatever you
% want to typeset, there is probably a package out there for it.
% Chemistry (chemtex), screenplays, you name it.
% Check out CTAN to see: http://www.ctan.org/
%%
\usepackage{graphicx,latexsym}
\usepackage{amsmath}
\usepackage{amssymb,amsthm}
\usepackage{longtable,booktabs,setspace}
\usepackage{chemarr} %% Useful for one reaction arrow, useless if you're not a chem major
\usepackage[hyphens]{url}
% Added by CII
\usepackage{hyperref}
\usepackage{lmodern}
\usepackage{float}
\floatplacement{figure}{H}
% End of CII addition
\usepackage{rotating}

% Next line commented out by CII
%%% \usepackage{natbib}
% Comment out the natbib line above and uncomment the following two lines to use the new
% biblatex-chicago style, for Chicago A. Also make some changes at the end where the
% bibliography is included.
%\usepackage{biblatex-chicago}
%\bibliography{thesis}


% Added by CII (Thanks, Hadley!)
% Use ref for internal links
\renewcommand{\hyperref}[2][???]{\autoref{#1}}
\def\chapterautorefname{Chapter}
\def\sectionautorefname{Section}
\def\subsectionautorefname{Subsection}
% End of CII addition

% Added by CII
\usepackage{caption}
\captionsetup{width=5in}
% End of CII addition

% \usepackage{times} % other fonts are available like times, bookman, charter, palatino

% Syntax highlighting #22

% To pass between YAML and LaTeX the dollar signs are added by CII
\title{My Final College Paper}
\author{Emerson H. Webb}
% The month and year that you submit your FINAL draft TO THE LIBRARY (May or December)
\date{May 2018}
\division{Mathematics and Natural Sciences}
\advisor{Advisor F. Name}
\institution{Reed College}
\degree{Bachelor of Arts}
%If you have two advisors for some reason, you can use the following
% Uncommented out by CII
% End of CII addition

%%% Remember to use the correct department!
\department{Mathematics}
% if you're writing a thesis in an interdisciplinary major,
% uncomment the line below and change the text as appropriate.
% check the Senior Handbook if unsure.
%\thedivisionof{The Established Interdisciplinary Committee for}
% if you want the approval page to say "Approved for the Committee",
% uncomment the next line
%\approvedforthe{Committee}

% Added by CII
%%% Copied from knitr
%% maxwidth is the original width if it's less than linewidth
%% otherwise use linewidth (to make sure the graphics do not exceed the margin)
\makeatletter
\def\maxwidth{ %
  \ifdim\Gin@nat@width>\linewidth
    \linewidth
  \else
    \Gin@nat@width
  \fi
}
\makeatother

\renewcommand{\contentsname}{Table of Contents}
% End of CII addition

\setlength{\parskip}{0pt}

% Added by CII

\providecommand{\tightlist}{%
  \setlength{\itemsep}{0pt}\setlength{\parskip}{0pt}}

\Acknowledgements{
I want to thank a few people.
}

\Dedication{
You can have a dedication here if you wish.
}

\Preface{
This is an example of a thesis setup to use the reed thesis document
class (for LaTeX) and the R bookdown package, in general.
}

\Abstract{
The preface pretty much says it all. \par

Second paragraph of abstract starts here.
}

	\usepackage{tikz, algorithm}
	\usepackage[noend]{algpseudocode}
% End of CII addition
%%
%% End Preamble
%%
%

\usepackage{amsthm}
\newtheorem{theorem}{Theorem}[chapter]
\newtheorem{lemma}{Lemma}[chapter]
\theoremstyle{definition}
\newtheorem{definition}{Definition}[chapter]
\newtheorem{corollary}{Corollary}[chapter]
\newtheorem{proposition}{Proposition}[chapter]
\theoremstyle{definition}
\newtheorem{example}{Example}[chapter]
\theoremstyle{definition}
\newtheorem{exercise}{Exercise}[chapter]
\theoremstyle{remark}
\newtheorem*{remark}{Remark}
\newtheorem*{solution}{Solution}
\begin{document}

% Everything below added by CII
  \maketitle

\frontmatter % this stuff will be roman-numbered
\pagestyle{empty} % this removes page numbers from the frontmatter
  \begin{acknowledgements}
    I want to thank a few people.
  \end{acknowledgements}
  \begin{preface}
    This is an example of a thesis setup to use the reed thesis document
    class (for LaTeX) and the R bookdown package, in general.
  \end{preface}
  \hypersetup{linkcolor=black}
  \setcounter{tocdepth}{2}
  \tableofcontents

  \listoftables

  \listoffigures
  \begin{abstract}
    The preface pretty much says it all. \par
    
    Second paragraph of abstract starts here.
  \end{abstract}
  \begin{dedication}
    You can have a dedication here if you wish.
  \end{dedication}
\mainmatter % here the regular arabic numbering starts
\pagestyle{fancyplain} % turns page numbering back on

\chapter{Delete line 6 if you only have one
advisor}\label{delete-line-6-if-you-only-have-one-advisor}

Placeholder

\chapter{Introduction to Trees, Random Forests, and the
Bootstrap}\label{rmd-basics}

Placeholder

\section{CART}\label{cart}

\section{Bootstrap}\label{bootstrap}

\section{Random Forests}\label{random-forests}

\subsection{How Bagged Forests Work}\label{how-bagged-forests-work}

\subsection{The Random Forest
Algorithm}\label{the-random-forest-algorithm}

\subsection{Variable Importance
Measures}\label{variable-importance-measures}

\subsection{Issues with Random
Forests}\label{issues-with-random-forests}

\section{Focus of this Thesis}\label{focus-of-this-thesis}

\section{Outline of Remaining
Chapters}\label{outline-of-remaining-chapters}

\chapter{Variable Importance and Inference for Random
Forests}\label{variable-importance-and-inference-for-random-forests}

\section{Introduction}\label{introduction}

In this chapter we focus on various approaches towards inference for
random forests that have been developed. Generally, we could consider to
be two approaches towards inference with random forests.

The first approach involves using variable importance measures of random
forests to evaluate the relative importance of different variables in
the construction of the forest. Ishwaran, Louppe, Owens, and Strobl et
al. have been involved in work in this area. Louppe and Ishwaran focused
on theoretical properties of variable importance measures while Owens,
and Strobl and colleagues work is centered on developing less biased
variable importance measures for hypothesis testing.

The second approach involves utilizing properties of the functional form
of the random forest estimator to construct prediction intervals. Once
predicton intervals have been constructed, depending on the context,
confidence intervals and hypothesis tests can be produced. Mentch and
Hooker's work involves interpreting the random forest ensemble as a
particular type of U-statistics, and applying theoretical results about
U-statistics to construct prediction intervals. Wager's work involves
applying the infintisemal jackknife to the random forest estimator to
produce prediction intervals. With both approaches there are asymptotic
results that we will discuss in some detail. In addition, we will
discuss some of the consistency results that have been proven for random
forests. It is worth noting here that in mathematically analyzing the
random forest algorithm, there are trade-offs between fidelity to the
CART algorithm and amenability to mathematical tool. To our best
knowledge, at this time there are no theoretical results showing that
random forests constructed using CART are consistent. Generally,
simplifications need to be made to the random forest algorithm to allow
for asymptotic analysis. These simplifications usually involve using a
different partitioning scheme than CART and also in working a sampling
without replacement framework. We will note when such assumptions are
being made when necessary. Otherwise, we assume a setting in which the
random forest is constructed using CART and bootstrap sampling with
replacement.

\section{Some Theory for Variable Importance
Measures}\label{some-theory-for-variable-importance-measures}

In this section we discuss Louppe and Ishwaran's work exploring
theoretical properties of variable importance measures.

\subsection{Ishwaran's Variable Importance
Measure}\label{ishwarans-variable-importance-measure}

Ishwaran's approach to variable importance measures of random forests,
as developed in ``Variable Importance in Binary Regression Trees and
Forests,'' differs from the original variable importance measures for
random forests, so we require some additional vocabulary.

Suppose \(T\) is a binary recursive tree and suppose that \(T\) has
\(M\) many terminal nodes. For each point \(\mathbf{x}\) in the feature
space, \(T\) maps \(\mathbf{x}\) to one of the \(M\)-many terminal
nodes. In particular, if we let \(\mathcal{X}\) denote the feature
space, then \(T\) is a function
\(\mathcal{X}\rightarrow \{1,\ldots, M\}\) defined by the equation
\[T(\mathbf{x})=\sum_{m=1}^M m B_m(\mathbf{x}),\] where
\(B_m(\mathbf{x})\) is a \(0-1\) basis function which partition the
feature space \(\mathcal{X}\).

Let \(Z=\{(\mathbf{x}_i,Y_i)|i=1,\ldots,n\}\) denote the training data,
where \(\mathbf{x}_i\) is a covariate in the feature space and \(Y_i\)
is the response. We call \(T\) a binary regression tree if it is a
binary recursive tree grown from \(Z\) using using binary recursive
splits of the form \(x_v\leq c\) and \(x_v> c\) where split values \(c\)
are chosen based on the observed \(\mathbf{x}_i\) in the training data
\(Z\). The value \(a_m\) in the terminal node is the average response of
the training observations falling in the \(m\)th node. That is,
\[a_m=\frac{\sum_{i=1}^n \mathbb{I}\{T(\mathbf{x}_i)=m\} Y_i}{\sum_{i=1}^n \mathbb{I}\{T(\mathbf{x}_i)=m\}}.\]

For a binary regression tree, the basis functions \(B_m(\mathbf{x})\)
are product splines of the following form:
\[B_m(\mathbf{x})=\prod_{l = 1}^{L_m} [x_{l(m)}-c_{l, m}]_{s_{l,m}},\]
where \(L_m\) denotes the number of splits used to construct
\(B_m(\mathbf{x})\). For each split \(l\), there is a splitting variable
\(\mathbf{x}_{l(m)}\) which denotes the \(l(m)\)th coordinate of
\(\mathbf{x}\) and a splitting value \(c_{l,m}\). The \(s_{l,m}\) are
binary \(\pm 1\) values, where for a given scalar \(x\),
\([x]_{+1}=\mathbb{I}(x>0)\) and \([x]_{-1}=\mathbb{I}(x\leq 0).\) Note
that the basis functions satisfy an orthogonality property, which gives
\(B_m(\mathbf{x})B_{m'}(\mathbf{x})=0\) if \(m\neq m'\). Note also that
given a tree \(T\), the predictor associated with the tree can be
written as a linear combination of basis functions:
\[\hat{\mu}(\mathbf{x})=\sum_{m=1}^M a_m B_m(\mathbf{x}).\] We are now
prepared to define Ishwaran's variable importance measure.

Informally, the \(MDA\) variable importance of a variable \(x_j\) is the
difference between \(MSE\) of the tree \(T\) when \(x_j\) is randomly
permuted and \(MSE\) of the tree \(T\) when \(x_j\) is not permuted. As
such a scheme of variable importance is difficult to analyze, Ishwaran
proposes a surrogate measure. For the variable \(x_j\), we drop
\(\mathbf{x}\) down the tree and follow the binary splits until either a
terminal node is reached or a node with a split depending on \(x_j\) is
reached. If a node with a split depending on \(x_j\) is reached, we then
subsequently assign \(\mathbf{x}\) randomly to either the left or right
daughter node, whenever there is a split, until we reach a terminal
node. The difference in \(MSE\) between noising up \(x_j\) and not
noising up \(x_j\) to be the variable importance of \(x_j\) in the tree
\(T\). Denote the tree that results from noising up \(x_j\) by \(T_j\).

Such a scheme relies on the following heuristic: if we chose an adequete
splitting rule to construct our tree, then we expect that variables that
are split earlier in the tree are more important, since prediction will
suffer the most from noising up a variable higher up in the tree than a
variable close to a terminal node. This is a behavior observed in CART
trees and random forests based on CART: splits closer to the root node
are more influential than splits close to terminal nodes, so the MDA or
MDI variable importance of variables split on close to the root node are
expected to be higher than otherwise.

\subsection{Maximal Subtrees and Theoretical
Results}\label{maximal-subtrees-and-theoretical-results}

Defining a structure on binary regression trees called subtrees,
Ishwaran is able to write the predictor for the noised up tree as a
deterministic component relying on terminal nodes for no parent nodes
involve a split on \(x_j\), and a random component involving terminal
nodes for which there are parent nodes involving a split on \(x_j\). The
definition of the subtree is quite intuitive. We call \(\tilde{T}_j\) a
\(j\)-subtree of the tree \(T\), if the root node of \(\tilde{T}_j\) has
daughters that depend on an \(x_j\) split. A \(j\)-subtree
\(\tilde{T}_j\) is a maximal \(j\)-subtree of the tree \(T\), if there
are no larger \(j\)-subtrees continaing \(\tilde{T}_j\). For a given
tree \(T\) and for each variable \(x_j\), there is a set of \(K_j\) many
distinct maximal \(j\)-subtrees, which are denoted by
\(\tilde{T}_{1,j},\ldots, \tilde{T}_{K_j,j}\). Note each distinct
\(\tilde{T}_{k,j}\) maximal \(j\)-subtree contains a set of distinct
terminal nodes \(M_{k,j}\). Each \(M_{k,j}\) is distinct for
\(k=1,\ldots,K_j\), since we are working with maximal \(j\)-subtrees.
Define \[M_j=\bigcup_{k=1}^{K_j} M_{k,j}\] to be the set of terminal
nodes for which there is a parent node involving a split on \(x_j\).
Ishwaran proves the following lemma about the functional form of the
predictor for the tree \(T_j\).
\begin{lemma}
Let $\hat{\mu}_j(\mathbf{x})$ denote the predictor for $T_j$. Then $$\hat{\mu}_j(\mathbf{x})=\sum_{m\notin M_j}a_m B_m(\mathbf{x})+\sum_{k=1}^{K_j} \tilde{a}_{k,j}\mathbb{I}\{T(\mathbf{x})\in M_{k,j}\},$$ where $\tilde{a}_{k,j}$ is the random terminal value assigned by $\tilde{T}_{k,j}$ under the random left right path through $\tilde{T}_{k,j}$. We write $\tilde{P}_{k,j}$ to denote the distribution of $\tilde{a}_{k,j}$. 
\end{lemma}
For a proof, the reader is referred to Ishwaran's paper. It is a bit
surprising that the functional form of \(\hat{mu}_j(\mathbf{x})\) can be
separated into the two components. Given this lemma and the definition
of \(j\)-subtrees, we can more formally define the variable importance
of \(x_j\) in the tree \(T\).

Let \(g\) be a loss function. Often we use the squared error to evaluate
loss, which corresponds to \(MSE\), but there is no strict requirement
in the definition. Denote the test data by \((Y,\mathbf{x})\). Then the
prediction error of the predictor \(\hat{\mu}\) is given by
\(\mathbb{E}(g(Y,\hat{\mu}(\mathbf{x})))\). As a reminder, we assume
that there is an underlying regression function
\[Y=\mu(\mathbf{x})+\varepsilon,\] where \(\varepsilon\) is independent
error with zero mean and variance \(\sigma^2>0\). Similarly, we can
define the prediction error of the predictor \(\hat{\mu}_j\) to be
\(\mathbb{E}(g(Y,\hat{\mu}_j(\mathbf{x}))).\) Set
\(g(Y,\hat{\mu}(\mathbf{x}))=(Y-\hat{\mu}(\mathbf{x}))^2\) to be the
\(L_2\) loss, which corresponds to \(MSE\). Define the variable
importance of the variable \(x_j\) to be
\[\Delta_j=\mathbb{E}((Y-\hat{\mu}_j(\mathbf{x}))^2)-\mathbb{E}((Y-\hat{\mu})^2).\]
Application of the lemma and some manipulation allows us to write
\[\Delta_j=\mathbb{E}(R_j(\mathbf{x})^2)-2\mathbb{E}\left(R_j(\mathbf{x})[\mu(\mathbf{x})-\hat{\mu}(\mathbf{x})]\right),\]
where
\[R_j(\mathbf{x})=\sum_{k=1}^{K_j}\sum_{m\in M_{k,j}} (\tilde{a}_{k,j} -a_m)B_m(\mathbf{x}).\]

To aid in his analysis, Ishwaran makes the assumption that the true
regression function \(\mu\) is of similar form to \(T\). That is, assume
\[\mu(\mathbf{x})=\sum_{m=1}^M a_{m,0} B_m(\mathbf{x}),\] where
\(a_{m,0}\) are the true, but unknown, terminal values. Under this and
some other large sample assumptions, Ishwaran finds that asymptotically,
each maximal \(j\)-subtree will tend to contribute equally to the
variable importance \(\Delta_j\). In effect, Ishwaran finds that nodes
closer to the root of a maximal \(j\)-subtree will have a larger effect
on \(\Delta_j\) than nodes closer to the terminal nodes.

\subsection{Extension to Forest
Ensembles}\label{extension-to-forest-ensembles}

Ishwaran's framework extends naturally to forest ensembles and his
theoretical result regarding forest ensembles provides some information
of the behavior of variable importance measures for random forests.
First for some notation, recall that in the forest ensemble setting, we
draw \(B\) many bootstrap resamples of the training data to obtain the
bootstrap replicates \(Z^{b}=\{(\mathbf{x}_i^b, Y_i^b)|i=1,\ldots,n\}\)
of the training data for \(b=1,\ldots, B\). We then construct a binary
regression tree \(T(\mathbf{x};b)\) on each bootstrap replicate of the
data and have the forest \(\hat{\mu}_F\) as the average of predictions
over the trees \(T(\mathbf{x};b)\):
\[\hat{\mu}_F(\mathbf{x})=\frac{1}{B}\sum_{b=1}^B \hat{\mu}(\mathbf{x};b),\]
where \(\hat{\mu}(\mathbf{x};b)\) denotes the predictor for the tree
\(T(\mathbf{x};b)\). Given that each \(\hat{\mu}(\mathbf{x};b)\) is a
linear combination of basis functions, we can write
\[\hat{\mu_F}(\mathbf{x})=\frac{1}{B}\sum_{b=1}^B\sum_{m=1}^{M^b}a_m^b B_m(\mathbf{x};b).\]
Again assume that the \(\mu(\mathbf{x})\) has a similar structure to
\(\mu(\mathbf{x})\). That is,
\(\mu(\mathbf{x})=\frac{1}{B}\sum_{b=1}^B\sum_{m=1}^{M^b}a_{m,0}^b B_m(\mathbf{x};b)\).

\chapter{The Bootstrap}\label{the-bootstrap}

\chapter*{Conclusion}\label{conclusion}
\addcontentsline{toc}{chapter}{Conclusion}

If we don't want Conclusion to have a chapter number next to it, we can
add the \texttt{\{-\}} attribute.

\textbf{More info}

And here's some other random info: the first paragraph after a chapter
title or section head \emph{shouldn't be} indented, because indents are
to tell the reader that you're starting a new paragraph. Since that's
obvious after a chapter or section title, proper typesetting doesn't add
an indent there.

\appendix

\chapter{The First Appendix}\label{the-first-appendix}

This first appendix includes all of the R chunks of code that were
hidden throughout the document (using the \texttt{include\ =\ FALSE}
chunk tag) to help with readibility and/or setup.

\textbf{In the main Rmd file}

\textbf{In Chapter \ref{ref-labels}:}

\chapter{The Second Appendix, for
Fun}\label{the-second-appendix-for-fun}

\chapter*{References}\label{references}
\addcontentsline{toc}{chapter}{References}

Placeholder


% Index?

\end{document}
